%Wiki https://ja.wikipedia.org/wiki/%E3%83%AA%E3%83%BC%E7%BE%A4

\documentclass{article}
\usepackage{amsmath}
\usepackage{amssymb}
\usepackage{amsfonts}
\usepackage{amsthm}
\usepackage{geometry}
\geometry{a4paper, margin=1in}

\newtheorem{definition}{定義} % ← 定義環境
\newtheorem{theorem}{定理} % ← 定理環境
\newtheorem{example}{例} % ← 例環境

\title{Lie群の概説}
\author{あなたの名前}
\date{\today}

\begin{document}

\maketitle

\section*{はじめに}
Lie群は、微分幾何学と群論が結びついた重要な概念であり、特に物理学や解析学において広く応用される。本稿では、Lie群の定義、基本的な性質、具体例、コンパクトLie群の分類、および表現論との関係について概説する。

\section{Lie群の定義}
\begin{definition}[Lie群]
$n$次元の\textbf{Lie群} $G$とは、次の条件を満たす集合である:
\begin{itemize}
    \item $G$は群の構造を持つ。
    \item $G$は$n$次元の滑らかな多様体の構造を持つ。
    \item 群演算(積$\mu: G \times G \to G$と逆元$\iota: G \to G, g \mapsto g^{-1}$)は$C^\infty$級である。
\end{itemize}
\end{definition}

\section{基本的な性質}
\subsection{Lie代数}
Lie群$G$の単位元$e$における接空間$\mathfrak{g} = T_e G$は、\textbf{Lie代数}と呼ばれる。Lie代数には次の性質がある:
\begin{itemize}
    \item $\mathfrak{g}$は$G$の局所的な構造を決定する。
    \item $\mathfrak{g}$には自然な\textbf{Lie括弧}(交換子ブラケット)$[\cdot, \cdot]: \mathfrak{g} \times \mathfrak{g} \to \mathfrak{g}$が定義される。
\end{itemize}

\subsection{指数写像}
Lie代数$\mathfrak{g}$からLie群$G$への写像
\[
\exp: \mathfrak{g} \to G
\]
は、Lie群の局所的な構造を解析するのに有用である。

\section{具体例}
\subsection{一般線形群 $GL(n, \mathbb{R})$}
$n$次正則行列全体の集合
\[
GL(n, \mathbb{R}) = \{ A \in M_n(\mathbb{R}) \mid \det A \neq 0 \}
\]
はLie群であり、Lie代数は$n$次正方行列全体の集合$\mathfrak{gl}(n, \mathbb{R}) = M_n(\mathbb{R})$である。

\subsection{特殊直交群 $SO(n)$}
\[
SO(n) = \{ A \in GL(n, \mathbb{R}) \mid A^T A = I, \det A = 1 \}
\]
は、回転群として知られ、Lie代数は
\[
\mathfrak{so}(n) = \{ X \in M_n(\mathbb{R}) \mid X^T = -X \}
\]
である。

\section{コンパクトLie群の分類}
コンパクトLie群は、次の3つのタイプに分類される:
\begin{itemize}
    \item \textbf{トーラス群} $T^n = (S^1)^n$:$n$次元の円周群
    \item \textbf{単純Lie群}:中心を除いて簡約できない群(例:$SU(n)$, $SO(n)$, $Sp(n)$)
    \item \textbf{有限被覆群}:単純Lie群の被覆群(例:$Spin(n)$)
\end{itemize}
コンパクトLie群の分類は、\textbf{Dynkin図}を用いて記述される。特に、単純Lie群は$A_n, B_n, C_n, D_n, E_6, E_7, E_8, F_4, G_2$の9種類の系列に分類される。

\section{Lie群の表現論}
Lie群の表現論は、群の作用を線形変換として解析する理論であり、次の基本概念を含む。

\subsection{既約表現}
\begin{definition}[既約表現]
Lie群$G$の\textbf{表現}とは、$G$からあるベクトル空間$V$への群準同型$\rho: G \to GL(V)$である。特に、$V$の非自明な部分空間がすべて$G$の作用で不変であるならば、$\rho$は\textbf{既約表現}と呼ばれる。
\end{definition}

\subsection{指標}
指標$\chi(g) = \operatorname{tr}(\rho(g))$は、表現の情報を圧縮した関数であり、表現の分類や分解に用いられる。

\subsection{重みとルート系}
Lie群$G$の表現は、対応するLie代数$\mathfrak{g}$の\textbf{重み}と\textbf{ルート系}によって特徴づけられる。例えば:
\begin{itemize}
    \item $SU(2)$の表現は、整数スピン$j$によって分類される。
    \item $SU(3)$の表現は、Young図形や重みラティスを用いて分類できる。
\end{itemize}

\section{重要な定理}
\begin{theorem}[Lieの第三定理]
任意の有限次元Lie代数$\mathfrak{g}$に対して、局所Lie群$G$が存在し、$\mathfrak{g}$が$G$のLie代数として実現される。
\end{theorem}

\begin{theorem}[Weylの指標公式]
コンパクトLie群$G$の既約表現の指標$\chi_\lambda$は、次のように与えられる:
\[
\chi_\lambda = \frac{\sum_{w \in W} \varepsilon(w) e^{w(\lambda + \rho) - \rho}}{\prod_{\alpha \in \Delta^+} (e^{\alpha/2} - e^{-\alpha/2})}
\]
ここで、$W$はワイル群、$\rho$は半整数和、$\Delta^+$は正のルートの集合である。
\end{theorem}

\section{まとめ}
本稿では、Lie群の定義から基本的な例、コンパクトLie群の分類、表現論との関係を概観した。特に、表現論において重みや指標が重要な役割を果たすことがわかった。

\end{document}
