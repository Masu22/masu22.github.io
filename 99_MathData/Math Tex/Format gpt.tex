%gpt(ring46964ring)で作成してもらったフォーマット文章

\documentclass{article}
\usepackage{amsmath}
\usepackage{amssymb}
\usepackage{amsfonts}
\usepackage{geometry}
\geometry{a4paper, margin=1in}

\title{数学概説}
\author{あなたの名前}
\date{\today}

\begin{document}

\maketitle

\section*{はじめに}
本書は、さまざまな数学のトピックについての簡単な概説を提供します。各項目は基本的な定義、定理、および例を含み、理解を深めることを目的としています。

\section{線形代数}
線形代数は、ベクトル空間、行列、線形変換などの概念を扱う数学の分野です。以下に主要なトピックを紹介します。

\subsection{ベクトル}
ベクトルは、方向と大きさを持つ量であり、通常は$\mathbb{R}^n$の空間で考えます。ベクトルの演算には加算、スカラー倍があり、内積や外積といった積の操作も重要です。

\[
\mathbf{v} = \begin{pmatrix} v_1 \\ v_2 \\ \vdots \\ v_n \end{pmatrix}
\]

\subsection{行列}
行列は、数値や変数を格子状に並べたもので、線形変換の表現や連立方程式の解法に用いられます。行列の乗法や逆行列についても重要な概念です。

\[
A = \begin{pmatrix}
a_{11} & a_{12} & \cdots & a_{1n} \\
a_{21} & a_{22} & \cdots & a_{2n} \\
\vdots & \vdots & \ddots & \vdots \\
a_{m1} & a_{m2} & \cdots & a_{mn}
\end{pmatrix}
\]

\subsection{行列の固有値と固有ベクトル}
行列$A$の固有値$\lambda$と固有ベクトル$\mathbf{v}$は、次のような関係を満たすベクトルとスカラーです。

\[
A \mathbf{v} = \lambda \mathbf{v}
\]

\section{微積分}
微積分は、関数の変化を扱う数学の分野で、微分と積分がその基礎です。

\subsection{微分}
微分は、関数の瞬間的な変化率を表す操作です。関数$f(x)$の微分は次のように定義されます。

\[
f'(x) = \lim_{\Delta x \to 0} \frac{f(x+\Delta x) - f(x)}{\Delta x}
\]

\subsection{積分}
積分は、関数の下の面積を求める操作で、逆に微分の操作を行います。定積分と不定積分があり、定積分は面積の計算に用いられます。

\[
\int_a^b f(x) \, dx
\]

\section{確率論}
確率論は、偶然の現象を数学的に扱う分野です。確率変数や確率分布などが重要な概念です。

\subsection{確率分布}
確率分布は、ある確率変数が取る可能性のある値と、それぞれの値の確率を示す関数です。例えば、二項分布や正規分布があります。

\[
P(X = x) = \frac{n!}{x!(n-x)!} p^x (1-p)^{n-x}
\]

\subsection{期待値}
確率変数$X$の期待値は、その確率分布の重み付き平均です。例えば、離散確率変数の場合、期待値は次のように表されます。

\[
E[X] = \sum_{i} x_i P(X = x_i)
\]

\end{document}
