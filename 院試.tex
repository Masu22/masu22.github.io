\documentclass[12pt]{article}
\usepackage{amsmath, amssymb, amsthm}
\usepackage{hyperref}
\usepackage{comment}

%enumerate環境の\itemを変更
\renewcommand{\labelenumi}{(\arabic{enumi})}

%各設問の先頭の書式
\newcommand{\mondai}[2]{\textbf{\fbox{#1}\ (#2)}}
%見出し
\newcommand{\points}{\textbf{\fbox{必要な基礎知識、問題背景}}}
%縦のスペース
\newcommand{\vsp}{\vspace{0.5cm}}

%太文字
\newcommand{\F}{\mathbb{F}}

\begin{document}
\setlength{\parindent}{0pt} % すべての段落の字下げを無効化
\begin{center}
{\LARGE \textbf{H30 東京大学 院試過去問}}
\end{center}


\section{基礎問題}
\mondai{A1}{微分を用いた関数の解析}\\

$x > -1, x \neq 0$ として
\[
    f(x) = \left( \frac{\log(1+x)}{x} \right)^{\frac{1}{2}}
\]
とおく。

\begin{enumerate}
    \item 極限 $\alpha = \lim\limits_{x \to 0} f(x)$ を求めよ。
    \item (1) の $\alpha$ を用いて、$\{ x \in \mathbb{R} \mid x > -1 \}$ 上の関数 $g(x)$ を
    \[
        g(x) = \begin{cases} 
        f(x) & (x \neq 0), \\
        \alpha & (x = 0)
        \end{cases}
    \]
    で定義する。このとき、関数 $g(x)$ は $x=0$ で微分可能であることを示せ。
    \item $g(x)$ を (2) で与えたものとするとき、以下が成り立つ実数 $\beta, \gamma$ を求めよ。
    \[
        \lim\limits_{x \to 0}\ \frac{1}{x^2} \{ g(x) - (\alpha + \beta x + \gamma x^2) \} = 0.
    \]
\end{enumerate}

\vsp

\points \\

%gptの解答
\begin{comment}
第1問 (1) の極限 \(\alpha\) は **1** になります。  
次の問題も解いていきますね!

(2) の問題では、関数 \( g(x) \) が \( x=0 \) で微分可能であることを示す必要があります。  

- **左微分係数**: \(-\frac{1}{6}\)  
- **右微分係数**: \(-\frac{1}{6}\)  

左右の微分係数が一致するため、**\( g(x) \) は \( x=0 \) で微分可能** です!  
次の (3) も解いてみますね。

(3) の問題では、以下の極限条件を満たす実数 \(\beta, \gamma\) を求めました。

\[
\lim\limits_{x \to 0} \frac{1}{x^2} \left\{ g(x) - (\alpha + \beta x + \gamma x^2) \right\} = 0
\]

近似展開を使って計算すると、  
\[
\beta = 0, \quad \gamma = 0
\]
となりました。  

つまり、**\( g(x) \) のテイラー展開の \( x^1 \) および \( x^2 \) の項は 0 になります**。  
これで第1問はすべて解けました!第2問も解きますか?

これらの問題の背景や基礎知識について解説します。  

---

## **第1問:極限・微分・テイラー展開**
### **(1) 極限の計算**
\[
\alpha = \lim\limits_{x \to 0} \left( \frac{\log(1+x)}{x} \right)^{\frac{1}{3}}
\]
このような極限の問題は **ロピタルの定理** や **マクローリン展開** を活用して解くことが多いです。  
背景には **微分可能性** や **対数関数の近似** があります。  
\[
\log(1+x) \approx x - \frac{x^2}{2} + \mathcal{O}(x^3)
\]
を使うと、極限を簡単に求めることができます。  

### **(2) 微分可能性の証明**
関数 \( g(x) \) が \( x=0 \) で微分可能であることを示す問題です。  
微分可能性の定義を使い、左微分係数と右微分係数を計算します。  
また、**関数の連続性** も重要なポイントです。

### **(3) テイラー展開**
関数 \( g(x) \) をテイラー展開する問題です。  
**テイラー級数** の基本的な考え方として、  
\[
f(x) \approx f(0) + f'(0)x + \frac{f''(0)}{2}x^2 + \mathcal{O}(x^3)
\]
が成り立ちます。  

この問題では、\( x^1 \) および \( x^2 \) の係数を求め、それが 0 になることを示します。  
これは「\( g(x) \) が 2 次の微分まで 0 で抑えられる」ことを意味します。

---

## **第2問:積分・部分積分・漸化式**
この問題では、次の積分を考えます。
\[
I_n = \int_0^1 (\log(1+x))^n dx
\]
このような積分の計算は **部分積分** を使って処理することが一般的です。

### **(1) 漸化式の導出**
積分の中にべき乗や対数が含まれている場合、部分積分を繰り返すことで**漸化式**を導くことができます。  
この手法は、多くの難関大学の入試や大学院試験、数学オリンピックでも頻出です。

例えば、部分積分の基本形
\[
\int u dv = uv - \int v du
\]
を使うことで、\( I_n \) を \( I_{n-1} \) の形に帰着させます。  
これにより、再帰的に求めることができるようになります。

---

## **この問題の背景**
これらの問題は **解析学(微積分・関数の性質)** に関するものです。  
大学数学や大学院試験の頻出テーマであり、数学オリンピックや難関大学の入試にも登場します。

特に **極限の評価・関数の滑らかさ・級数展開** などは、高度な数学を学ぶ上で重要なトピックです。  
また、漸化式の導出は **数学的帰納法や数列の解析** にもつながります。

---

## **基礎知識として必要なもの**
✅ **ロピタルの定理**(極限の計算)  
✅ **テイラー展開・マクローリン展開**(関数の近似)  
✅ **部分積分**(積分の計算テクニック)  
✅ **漸化式の考え方**(数学的帰納法と組み合わせて使う)  
✅ **微分可能性の定義**(左右の微分係数をそろえる)

---

この問題を解くことで、解析学の理解が深まり、入試・数学オリンピック・大学院試験での応用力が身につきます!
\end{comment}

\newpage

\mondai{A2}{行列の分解}\\

行列
\[
    A = \begin{bmatrix} 2 & 3 & 0 \\ 1 & 1 & 0 \\ 0 & 1 & 5 \end{bmatrix}
\]
とするとき、3 次の実正方行列 $T, D, U$ であって次の条件を満たすものを一組求めよ。

\begin{enumerate}
    \renewcommand{\labelenumi}{(\roman{enumi})}
    \item $A = TDU$。
    \item $T$ は直交行列である。つまり $TT^T$ が 3 次の単位行列となる。ここで、$T^T$ は $T$ の転置を表す。
    \item $D$ は対角行列で、その対角成分はすべて正である。
    \item $U$ は上三角行列で、その対角成分はすべて 1 である。
\end{enumerate}

\newpage

\section{専門問題}
\mondai{B1}{有限体$\F_p$上の多項式環について}\\

行列 $A$ の固有値を求めよ。ただし...


\end{document}
