\documentclass{article}
\usepackage{amsmath, amssymb, amsthm}
\usepackage{geometry}
\geometry{a4paper, margin=1in}

\begin{document}

\title{令和7年度 京都大学大学院理学研究科 数学・数理解析専攻\newline 数学系・数理解析系 入学試験問題}
\author{}
\date{2025}
\maketitle

\section*{専門科目 Advanced Mathematics}

\subsection*{試験情報}
\begin{itemize}
    \item 問題は13題あり、数学系志望者は1〜11のうちの2題を選択して解答する。ただし、9と10の2題を同時に選択してはならない。
    \item 数理解析系志望者は1〜13のうちの2題を選択して解答する。
    \item 解答時間は2時間30分。
    \item 参考書・ノート類・電卓・携帯電話・情報機器・時計等の持ち込みは禁止。
\end{itemize}

\section*{問題}
\textbf{1. 次の条件を満たす群 $G$ をすべて求めよ。}
\begin{itemize}
    \item (条件1) $G$ は有限アーベル群である。
    \item (条件2) $G$ は $SL_2(\mathbb{Z})$ の部分群と同型である。
\end{itemize}

\textbf{2.} $\mathbb{R}$ 上の 2 変数多項式環 $\mathbb{R}[X,Y]$ の極大イデアルは、次のいずれかの形で表されることを示せ。
\begin{itemize}
    \item $(X + a, Y + b)$ ($a, b \in \mathbb{R}$)
    \item $(X^2 + aX + b, Y + cX + d)$ ($a, b, c, d \in \mathbb{R}$)
    \item $(X + aY + b, Y^2 + cY + d)$ ($a, b, c, d \in \mathbb{R}$)
\end{itemize}

\textbf{3.} 正整数 $n$ に対し整数を係数とする多項式 $f_n(X)$ を次のように定める。
\begin{align*}
    f_1(X) &= X^2 - 2, \\
    f_n(X) &= f_{n-1}(X^2 - 2), \quad (n \geq 2)
\end{align*}
$K_n$ を $f_n(X)$ の $\mathbb{Q}$ 上の最小分解体とする。このとき、ガロア群 $\text{Gal}(K_n/\mathbb{Q})$ を求めよ。

\end{document}
