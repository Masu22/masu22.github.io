%Lualatexで実行すること!contentsのために、2回コンパイルする必要あり

\documentclass[a4paper, 12pt]{article}

\title{数学の分野ごとのまとめ}
\author{あなたの名前}
\date{\today}

\usepackage[svgnames]{xcolor} %色を使うためのパッケージです。svgnames オプションにより、SVGの色名を利用できます。
\usepackage{amsmath, amsthm, amssymb} %数式や定理環境を使うためのパッケージです。
\usepackage{thmtools} %定理環境のカスタマイズをより簡単にするためのパッケージです。
\usepackage{lipsum} %ダミーテキスト(Lorem Ipsum)を生成するパッケージ。実際の内容に入れる前に仮のテキストを挿入するのに使用。
\usepackage{framed} %囲み枠を使うためのパッケージです。
\usepackage{luatexja}

\usepackage{hyperref}


%headcolour と rulecolour に色を設定
\colorlet{headcolour}{DarkOrange}
\colorlet{rulecolour}{DarkOrange}

%--------------------------------
%leftbar 環境のカスタマイズ:
%\def\FrameCommand で、囲み枠の左側に縦線(幅3pt)を描画し、その後に10ptのスペースを空けるように設定。
%\MakeFramed と \endMakeFramed で囲み枠を作成し、枠の中身が正しく表示されるようにします。
\renewenvironment{leftbar}{
  \def\FrameCommand{{\color{rulecolour}\vrule width 3pt} \hspace{10pt}}
  \MakeFramed {\advance\hsize-\width \FrameRestore}}
 {\endMakeFramed}

%-------------------------------
%新しい定理環境のスタイルを定義
 \declaretheoremstyle[headfont=\sffamily\bfseries, %ヘッダのフォントを設定
 notefont=\sffamily\bfseries, %定理の番号(NOTE)のフォント設定
 notebraces={}{}, %定理番号の周りに括弧を表示しない設定。
 headpunct=, %ヘッダーの後に句読点をつけない設定
 bodyfont=\sffamily\itshape, %本文のフォントをサンセリフ体のイタリック(itshape)に設定
 headformat=\color{headcolour}\NAME~\NUMBER\NOTE\hfill\smallskip\linebreak, % ヘッダーにタイトル(NAME)と番号(NUMBER)を表示し、その後に定理番号(NOTE)を配置します。色は headcolour に設定されたオレンジ色です。
 preheadhook=\begin{leftbar}, %定理のタイトルの前に囲み枠(leftbar)を挿入
 postfoothook=\end{leftbar},  %定理のタイトルの後に囲み枠の終了を挿入
 ]{customDefinition}
 \declaretheorem[style=customDefinition, numberwithin=section]{definition} %環境名definitionで\NAMEはDefinitionになる。
%他にも、環境名を変えると\NAMEが対応して変わる!!
%| 環境名          | 表示される名前(`\NAME`) |
%| ------------ | ---------------- |
%| `theorem`    | Theorem          |
%| `definition` | Definition       |
%| `lemma`      | Lemma            |
%| `corollary`  | Corollary        |
%| `example`    | Example          |
%自分で名前を付けたいときは、nameオプションを使う。例えば
%\declaretheorem[style=customDefinition, numberwithin=section, name=定義]{definition}
%-------------------------------

% 定義環境と定理環境の設定  ( 上記の色付きleftbar環境とは別に定義 )
\newtheorem{defi}{定義}[section]  % セクションごとに番号が振られる
\newtheorem{theorem}{定理}[section]     % セクションごとに番号が振られる
\newtheorem{lemma}{補題}[section]       % セクションごとに番号が振られる
\newtheorem{corollary}{系}[section]     % セクションごとに番号が振られる
\newtheorem{problem}{問題}[section]
\newtheorem{solution}{解法}[section]

\begin{document}

\maketitle
\tableofcontents  % 目次の自動生成

\newpage

\section{色付きのleftbar環境の例}

%leftbar環境の出力のため、カウンターを1にセットした
\setcounter{section}{1}

\begin{definition}[(Matrix)]
\lipsum[2]\\
↑はダミーテキスト
\end{definition}

\noindent こんな感じで、左線付きの環境ができた!!

\begin{definition}
本文はここ。こちらはタイトルの補足がないバージョン
\end{definition}

\section{前提知識 (Preliminary Knowledge)}
\subsection{定義 (Definitions)}
% ここに定義を追加します
\begin{defi}
    数列 $a_n$ が収束するとは、$\lim_{n \to \infty} a_n = L$ となることです。
\end{defi}

\subsection{定理 (Theorems)}
% ここに定理を追加します
\begin{theorem}
    $\lim_{n \to \infty} a_n = L$ が成り立つならば、数列 $a_n$ は収束すると言います。
\end{theorem}

\section{定理と証明 (Theorems and Proofs)}
\begin{theorem}[例: 関数の極限]
    $f(x)$ が $x_0$ で連続ならば、$\lim_{x \to x_0} f(x) = f(x_0)$。
\end{theorem}

\begin{proof}
    証明の詳細。
\end{proof}

\section{例 (Examples)}
\subsection{例1: 数列の収束}
数列 $a_n = \frac{1}{n}$ は $0$ に収束する。

\subsection{例2: 関数の極限}
関数 $f(x) = x^2$ は $x_0 = 2$ で連続であり、$\lim_{x \to 2} f(x) = 4$。

\section{補題 (Lemmas)}
\begin{lemma}
    任意の実数列 $a_n$ が上に有界ならば、部分列の中で収束するものが存在する。
\end{lemma}

\begin{proof}
    補題の証明。
\end{proof}

\section{問題 (Problems)}
\begin{problem}
    数列 $a_n = \frac{2n+1}{n^2 + 1}$ の極限を求めよ。
\end{problem}

\begin{solution}
    解答の詳細。
\end{solution}

\section{文献 (References)}
\begin{thebibliography}{99}
    \bibitem{example1} 書籍名, 著者, 出版年.
    \bibitem{example2} 論文名, 著者, 雑誌名, 年.
\end{thebibliography}

\appendix
\section{付録 (Appendix)}
ここには計算の詳細や補足説明を追加します。
\end{document} 
