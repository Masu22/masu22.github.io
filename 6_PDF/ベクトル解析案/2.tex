\documentclass[a4,11pt]{article}
\usepackage{luatexja}
\usepackage{amsmath,amssymb}
\usepackage{graphicx}
\usepackage{tikz}
\usepackage{geometry}
\geometry{margin=25mm}

\title{第2回:発散(div)とガウスの定理}
\author{}
\date{}

\begin{document}
\maketitle

\section*{1. 発散(divergence)とは?}

\subsection*{1.1 ベクトル場の「広がり具合」}
ベクトル場 $\vec{F}(x, y) = (P(x, y), Q(x, y))$ の発散とは:

\[
\mathrm{div} \, \vec{F} = \frac{\partial P}{\partial x} + \frac{\partial Q}{\partial y}
\]

\begin{itemize}
  \item ベクトル場が「その点から湧き出しているか」を示す。
  \item 水源・風・電荷の分布などのモデルに使える。
\end{itemize}

\subsection*{1.2 例:}
\[
\vec{F}(x, y) = (x, y) \Rightarrow \mathrm{div} \, \vec{F} = \frac{\partial x}{\partial x} + \frac{\partial y}{\partial y} = 1 + 1 = 2
\]

\begin{center}
\begin{tikzpicture}[scale=1]
  \draw[->] (-3,0) -- (3,0) node[right] {$x$};
  \draw[->] (0,-3) -- (0,3) node[above] {$y$};
  \foreach \x in {-2,-1,0,1,2} {
    \foreach \y in {-2,-1,0,1,2} {
      \draw[->, blue] (\x,\y) -- ++(0.2*\x,0.2*\y);
    }
  }
  \node at (0,-3.5) {図:$\vec{F}(x, y) = (x, y)$ の発散};
\end{tikzpicture}
\end{center}

\subsection*{1.3 物理的イメージ}
\begin{itemize}
  \item 発散が正:その点から流れが湧き出している(源)。
  \item 発散が負:その点に流れが吸い込まれている(吸収)。
  \item 発散が0:流れは保存されている(例: incompressible な流体)。
\end{itemize}

\section*{2. ガウスの定理(発散定理)}

\subsection*{2.1 内容}
閉曲線に囲まれた領域 $D$ の内部の発散の総和は、境界を通過する流れの総量に等しい:

\[
\iint_D \mathrm{div} \, \vec{F} \, dA = \oint_{\partial D} \vec{F} \cdot \vec{n} \, ds
\]

\begin{itemize}
  \item 左辺:内部の「湧き出し量」の総和
  \item 右辺:外に出ていく流れ(境界線積分)
\end{itemize}

\subsection*{2.2 図での理解}

\begin{center}
\begin{tikzpicture}[scale=1.2]
  \draw[thick] (0,0) circle(2);
  \node at (0,0) {$D$};
  \draw[->, thick, red] (2,0) -- (2.5,0); 
  \node at (2.7,0) {$\vec{n}$};
  \node at (0,-2.5) {図:ガウスの定理のイメージ};
\end{tikzpicture}
\end{center}

\subsection*{2.3 応用例}
\begin{itemize}
  \item 電場のガウスの法則:電荷があると電場が湧き出す。
  \item 流体力学:密度保存されていれば発散は0。
\end{itemize}

\subsection*{2.4 例題}
\[
\vec{F}(x, y) = (x, y) \quad \text{を} \quad D = \{(x, y) \mid x^2 + y^2 \leq 1\} \text{で考える}
\]

発散:$\mathrm{div} \, \vec{F} = 2$

面積:$|D| = \pi$

左辺:
\[
\iint_D \mathrm{div} \, \vec{F} \, dA = \iint_D 2 \, dA = 2 \pi
\]

右辺(境界積分)も計算可能だが、ここでは省略。  
→ \textbf{内部の発散の総和 = 境界から出ていく流れの総和}

\end{document}