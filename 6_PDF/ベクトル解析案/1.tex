\documentclass[a4j,11pt]{jsarticle}
\usepackage{amsmath,amssymb}
\usepackage{graphicx}
\usepackage{tikz}
\usepackage{geometry}
\geometry{margin=25mm}

\title{第1回:ベクトル解析入門}
\author{}
\date{}

\begin{document}
\maketitle

\section*{1. ベクトルとは何か?(復習)}

\subsection*{1.1 ベクトルの基本}
ベクトルとは、\textbf{大きさと向きをもつ量}のこと。

\begin{itemize}
  \item 記号:$\vec{a}, \vec{b}$ など
  \item 成分表示:$\vec{a} = (a_1, a_2)$ や $\vec{a} = a_1\vec{i} + a_2\vec{j}$
\end{itemize}

\subsection*{1.2 ベクトルの演算}
\begin{itemize}
  \item 加法:$\vec{a} + \vec{b}$(並べて結ぶ)
  \item スカラー倍:$k\vec{a}$
  \item 内積:$\vec{a} \cdot \vec{b} = |\vec{a}||\vec{b}|\cos\theta$
\end{itemize}

\subsection*{1.3 空間内のベクトル}
3次元ベクトル:$\vec{a} = (a_1, a_2, a_3)$  
基底ベクトル $\vec{i}, \vec{j}, \vec{k}$ を使って表す。

\[
\vec{a} = a_1\vec{i} + a_2\vec{j} + a_3\vec{k}
\]

\section*{2. ベクトル場のイメージ}

\subsection*{2.1 ベクトル場とは?}
\textbf{空間の各点にベクトルが割り当てられているもの}。  
例:風の流れ場、電場、磁場、流体の速度場など。

\begin{center}
\begin{tikzpicture}[scale=1]
  \draw[->] (-3,0) -- (3,0) node[right] {$x$};
  \draw[->] (0,-2) -- (0,2) node[above] {$y$};
  \foreach \x in {-2,-1,0,1,2} {
    \foreach \y in {-1,0,1} {
      \draw[->, blue] (\x,\y) -- ++(0.2*\x,0.2*\y);
    }
  }
  \node at (0,-2.5) {図:原点から外向きに広がるベクトル場};
\end{tikzpicture}
\end{center}

\subsection*{2.2 数式での定義}
2次元ベクトル場:$\vec{F}(x, y) = (P(x, y), Q(x, y))$  
3次元:$\vec{F}(x, y, z) = (P, Q, R)$

\vspace{3mm}
例:
\[
\vec{F}(x, y) = (x, y) \quad \text{→ 原点から外向きに広がる}
\]

\section*{3. スカラー場と勾配(grad)}

\subsection*{3.1 スカラー場とは?}
スカラー場とは、\textbf{空間の各点に「数値(スカラー)」が対応しているもの}。

\begin{itemize}
  \item 温度分布:$T(x, y)$(場所ごとに温度が決まる)
  \item 標高:$h(x, y)$(地図上の各点の高さ)
\end{itemize}

\textbf{例:}  
\[
f(x, y) = x^2 + y^2 \quad \Rightarrow \text{原点から離れるほど数値が大きい}
\]

\subsection*{3.2 等高線と勾配ベクトル}

\textbf{等高線:} $f(x, y) = c$ の形で表される「同じ値の点の集合」

\textbf{勾配ベクトル:} スカラー場 $f(x, y)$ に対して
\[
\nabla f = \left( \frac{\partial f}{\partial x}, \frac{\partial f}{\partial y} \right)
\]
というベクトル場を定義する。

\begin{itemize}
  \item 勾配ベクトルは「関数の値が最も速く増加する方向」
  \item かつ、「等高線に垂直」
\end{itemize}

\begin{center}
\begin{tikzpicture}[scale=1.2]
  \draw[->] (-3,0) -- (3,0) node[right] {$x$};
  \draw[->] (0,-2) -- (0,2) node[above] {$y$};

  \foreach \r in {0.5,1.0,1.5} {
    \draw[gray, thick] (0,0) circle[radius=\r];
  }
  \draw[->, red, thick] (1.0,0) -- (1.4,0);
  \node[right] at (1.4,0) {$\nabla f$};
  \node at (2.2,1.8) {等高線:$f(x,y) = x^2 + y^2 = c$};
\end{tikzpicture}
\end{center}

\subsection*{3.3 勾配の物理的イメージ}

\textbf{坂道を登るとき、どの方向が一番きつい?}  
→ それが「勾配ベクトルの方向」!

\begin{itemize}
  \item 数学的には $\nabla f$ の方向
  \item 物理的には「力が自然に働く方向」
\end{itemize}

\subsection*{3.4 例題:}
関数 $f(x, y) = x^2 + y^2$ の勾配を求めよ。

\[
\nabla f = \left( \frac{\partial f}{\partial x}, \frac{\partial f}{\partial y} \right) = (2x, 2y)
\]

原点では $\nabla f = (0, 0)$  
→ 平らな場所(傾きがない)

点 $(1, 0)$ では $\nabla f = (2, 0)$  
→ x方向に登りがきつい!

\section*{4. 線形変換とヤコビ行列への導入}

\subsection*{4.1 関数の線形近似とは?}

2変数関数 $f(x, y)$ の微分は、「接平面での近似」を意味する。

点 $(a, b)$ における近似:
\[
f(x, y) \approx f(a, b) + \frac{\partial f}{\partial x}(a, b)(x - a) + \frac{\partial f}{\partial y}(a, b)(y - b)
\]

\subsection*{4.2 ヤコビ行列の定義}

$f: \mathbb{R}^n \to \mathbb{R}^m$ のとき、ヤコビ行列(Jacobian matrix)は:

\[
J_f(x) = 
\begin{bmatrix}
\displaystyle \frac{\partial f_1}{\partial x_1} & \cdots & \displaystyle \frac{\partial f_1}{\partial x_n} \\
\vdots & \ddots & \vdots \\
\displaystyle \frac{\partial f_m}{\partial x_1} & \cdots & \displaystyle \frac{\partial f_m}{\partial x_n}
\end{bmatrix}
\]

2変数ベクトル関数 $\vec{F}(x, y) = (P(x, y), Q(x, y))$ の場合:

\[
J_{\vec{F}}(x, y) = 
\begin{bmatrix}
\displaystyle \frac{\partial P}{\partial x} & \displaystyle \frac{\partial P}{\partial y} \\
\displaystyle \frac{\partial Q}{\partial x} & \displaystyle \frac{\partial Q}{\partial y}
\end{bmatrix}
\]

\subsection*{4.3 例題:}
\[
\vec{F}(x, y) = (x^2 - y, x + y^2)
\]

ヤコビ行列:
\[
J_{\vec{F}}(x, y) =
\begin{bmatrix}
2x & -1 \\
1 & 2y
\end{bmatrix}
\]

\end{document}