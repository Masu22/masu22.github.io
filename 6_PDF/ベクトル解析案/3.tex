\documentclass[a4j,11pt]{jsarticle}
\usepackage{amsmath,amssymb}
\usepackage{graphicx}
\usepackage{tikz}
\usepackage{geometry}
\geometry{margin=25mm}

\title{第3回:カール(curl)とストークスの定理}
\author{}
\date{}

\begin{document}
\maketitle

\section*{1. カール(curl)とは?}

\subsection*{1.1 ベクトル場の「回転の強さ」}
ベクトル場 $\vec{F}(x, y) = (P(x, y), Q(x, y))$ に対して、

\[
\mathrm{curl} \, \vec{F} = \frac{\partial Q}{\partial x} - \frac{\partial P}{\partial y}
\]

\begin{itemize}
  \item 2次元の場合、スカラー量(回転の強さ)になる。
  \item 3次元では $\nabla \times \vec{F}$ のようにベクトルになる。
\end{itemize}

\subsection*{1.2 例}
\[
\vec{F}(x, y) = (-y, x) \Rightarrow \mathrm{curl} \, \vec{F} = \frac{\partial x}{\partial x} - \frac{\partial (-y)}{\partial y} = 1 + 1 = 2
\]

\begin{center}
\begin{tikzpicture}[scale=1]
  \draw[->] (-3,0) -- (3,0) node[right] {$x$};
  \draw[->] (0,-3) -- (0,3) node[above] {$y$};
  \foreach \x in {-2,-1,0,1,2} {
    \foreach \y in {-2,-1,0,1,2} {
      \draw[->, blue] (\x,\y) -- ++(-0.2*\y, 0.2*\x);
    }
  }
  \node at (0,-3.5) {図:$\vec{F}(x, y) = (-y, x)$ のカール};
\end{tikzpicture}
\end{center}

\subsection*{1.3 物理的イメージ}
\begin{itemize}
  \item カールが正:その点のまわりに反時計回りの回転。
  \item カールが負:時計回りの回転。
  \item カールが0:その点のまわりに回転がない。
\end{itemize}

\section*{2. ストークスの定理}

\subsection*{2.1 内容}
曲面 $S$ の上の回転の総和は、境界を一周したときの流れに等しい:

\[
\iint_S (\nabla \times \vec{F}) \cdot \vec{n} \, dS = \oint_{\partial S} \vec{F} \cdot d\vec{r}
\]

\begin{itemize}
  \item 左辺:面積分(回転ベクトルと法線ベクトルの内積)
  \item 右辺:線積分(境界を一周した流れ)
\end{itemize}

\subsection*{2.2 図での理解}

\begin{center}
\begin{tikzpicture}[scale=1.2]
  \draw[thick] (0,0) circle(2);
  \node at (0,0) {$S$};
  \draw[->, thick, red] (0,0) -- (0,2.5); 
  \node at (0.3,2.7) {$\vec{n}$};
  \draw[->, thick] (2,0) arc (0:90:2);
  \node at (1.3,1.3) {$\partial S$};
  \node at (0,-3) {図:ストークスの定理のイメージ};
\end{tikzpicture}
\end{center}

\subsection*{2.3 応用例}
\begin{itemize}
  \item 電磁気学:ファラデーの法則
  \item 流体:渦の強さの計算
\end{itemize}

\subsection*{2.4 例題}
\[
\vec{F}(x, y) = (-y, x), \quad S = \{(x, y) \mid x^2 + y^2 \leq 1\}
\]

カール:$\mathrm{curl} \, \vec{F} = 2$

面積:$|S| = \pi$

左辺:
\[
\iint_S 2 \, dA = 2\pi
\]

右辺(線積分)も計算可能:円周を一周する線積分でも $2\pi$。

\textbf{結論:ストークスの定理成立!}

\end{document}