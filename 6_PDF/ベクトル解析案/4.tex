\documentclass[a4j,11pt]{jsarticle}
\usepackage{amsmath,amssymb}
\usepackage{graphicx}
\usepackage{tikz}
\usepackage{geometry}
\geometry{margin=25mm}

\title{第4回:微分形式とストークスの定理の正体}
\author{}
\date{}

\begin{document}
\maketitle

\section*{1. 微分形式ってなに?}

\subsection*{1.1 高校数学とのつながり}
高校の「$dx$, $dy$」の記号は、実は「微分形式」と呼ばれるもの。

\[
dx, \quad dy \quad \text{は 1-形式と呼ばれる}
\]

例:ベクトル場 $\vec{F} = (P, Q)$ に対して、対応する 1-形式は

\[
\omega = P(x, y) \, dx + Q(x, y) \, dy
\]

\subsection*{1.2 微分形式に積分できる}
\[
\int_C \omega = \int_C P \, dx + Q \, dy
\]

→ これが線積分の正体!

\section*{2. 外微分と $d\omega$}

\subsection*{2.1 外微分の定義}
1-形式 $\omega = P \, dx + Q \, dy$ の外微分は

\[
d\omega = \left( \frac{\partial Q}{\partial x} - \frac{\partial P}{\partial y} \right) dx \wedge dy
\]

ここで $\wedge$ は「外積(くさび積)」で、面積を表す形式。

→ 実は、これは curl の正体!

\subsection*{2.2 線積分 vs 面積分}
\[
\text{線積分:} \quad \int_{\partial D} \omega
\]
\[
\text{面積分:} \quad \iint_D d\omega
\]

\section*{3. 一般化されたストークスの定理}

\subsection*{3.1 ストークスの定理(微分形式ver.)}
任意の微分形式 $\omega$ に対して:

\[
\int_{\partial M} \omega = \int_M d\omega
\]

\begin{itemize}
  \item $\omega$: $k$-形式
  \item $d\omega$: $(k+1)$-形式
  \item $M$: $k+1$ 次元多様体
  \item $\partial M$: 境界
\end{itemize}

\subsection*{3.2 図で理解しよう}

\begin{center}
\begin{tikzpicture}[scale=1.2]
  \draw[thick, fill=blue!10] (0,0) circle(2);
  \draw[thick, ->] (2,0) arc (0:90:2);
  \node at (1.3,1.3) {$\partial M$};
  \node at (0,0) {$M$};
  \node at (0,-2.5) {図:微分形式のストークス定理};
\end{tikzpicture}
\end{center}

\subsection*{3.3 特別な場合としてのガウス・ストークス}
\begin{itemize}
  \item ベクトル解析のストークスの定理は、この一般形の2次元バージョン。
  \item ガウスの定理は、3次元の発散(=外微分)に関する場合。
\end{itemize}

\section*{4. まとめ}

\begin{itemize}
  \item $dx, dy$ はただの記号じゃない!
  \item 微分形式を使うと、線積分も面積分も同じ枠組みで語れる!
  \item 「一般のストークス定理」はすべてを統一する美しい定理!
\end{itemize}

\end{document}