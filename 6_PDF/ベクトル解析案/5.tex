\documentclass[a4j,11pt]{jsarticle}
\usepackage{amsmath,amssymb}
\usepackage{graphicx}
\usepackage{tikz}
\usepackage{geometry}
\geometry{margin=25mm}

\title{第5回:ベクトル解析の統一的視点と応用}
\author{}
\date{}

\begin{document}
\maketitle

\section*{1. 微分形式による物理法則の記述}

\subsection*{1.1 マクスウェル方程式の微分形式}
電磁気学の基本法則であるマクスウェル方程式は、微分形式を用いると美しく表現できます。例えば、\textbf{ガウスの法則}は次のように記述できます。

\[
d\omega_E = \rho \, dx \wedge dy \quad \text{(電場の源としての電荷密度)}
\]

\subsection*{1.2 磁場とストークスの定理}
\[
d\omega_B = 0 \quad \text{(磁場には源がない)}
\]

ストークスの定理を使って、磁場が閉じた曲線上で循環する様子を記述することも可能です。

\[
\oint_{\partial S} \vec{B} \cdot d\vec{r} = \iint_S (\nabla \times \vec{B}) \cdot d\vec{A}
\]

\section*{2. 幾何学的理解:多様体上のベクトル解析}

\subsection*{2.1 微分形式と多様体}
ベクトル解析は、実は多様体上での微分幾何学の一部として捉えることができます。多様体上での微分形式は、ある種の「局所的な」物理法則を記述するために使われます。

\subsection*{2.2 例:円環上のベクトル場}
例えば、円環上のベクトル場を微分形式で表現する場合、$dx$ や $dy$ のような微分形式を使うことで、ベクトル場の局所的な挙動を記述できます。

\begin{center}
\begin{tikzpicture}[scale=1.2]
  \draw[thick] (0,0) circle(2);
  \node at (0,0) {$M$};
  \node at (0,-2.5) {図:円環上のベクトル場};
\end{tikzpicture}
\end{center}

\section*{3. ベクトル場と保存量}

\subsection*{3.1 保存則の美しさ}
ベクトル場における保存則は、ガウスの定理やストークスの定理の美しい帰結として現れます。例えば、保存ベクトル場に対する線積分は、その経路に依存しない、すなわち積分値が定まることが知られています。

\subsection*{3.2 例:電場の保存性}
電場 $\vec{E}$ が保存ベクトル場であれば、次のような式が成立します。

\[
\oint_{\partial S} \vec{E} \cdot d\vec{r} = 0
\]

\section*{4. ベクトル解析の三大定理の比較}

\subsection*{4.1 ストークス・ガウス・グリーンの定理の関係}
ベクトル解析の三大定理であるストークスの定理、ガウスの定理、グリーンの定理は、すべて「境界と中身の関係」に基づいています。これらの定理は微分形式を使うことで統一的に理解できます。

\subsection*{4.2 三大定理の図示}
\begin{center}
\begin{tikzpicture}[scale=1.2]
  \draw[thick, fill=blue!10] (0,0) circle(2);
  \node at (0,0) {$S$};
  \draw[->, thick] (2,0) arc (0:90:2);
  \node at (1.3,1.3) {$\partial S$};
  \node at (0,-2.5) {図:ベクトル解析の三大定理の関係};
\end{tikzpicture}
\end{center}

\section*{5. まとめ}

\begin{itemize}
  \item 微分形式を使うと、物理法則が一貫して記述できる。
  \item ベクトル解析の三大定理は、境界と中身の関係に関わる。
  \item 微分形式によって、ベクトル場の保存則や循環を美しく理解できる。
\end{itemize}

\end{document}