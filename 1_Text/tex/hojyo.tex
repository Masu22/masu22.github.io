%lualatexで実行すること!!

\documentclass[a4paper,12pt]{article}
\usepackage[margin=25mm]{geometry}
\usepackage{array}
\usepackage{tikz}
\usepackage{graphicx}
\usepackage{titlesec}
\usepackage{booktabs}
\usepackage{luatexja}

\titleformat{\section}{\large\bfseries}{\thesection.}{1em}{}

\usetikzlibrary{positioning}

\begin{document}

\begin{center}
    {\LARGE \bf 相談補助資料}
\end{center}

\vspace{3mm}

%=============================
\section{簡易収支予想(例:月間ベース)}
\vspace{2mm}
\begin{tabular}{|p{8cm}|r|}
  \hline
  \textbf{項目} & \textbf{金額(円)} \\
  \hline
  売上(例:月100件 × 単価3,000円) & 300,000 \\
  \hline
  仕入・材料費 & 60,000 \\
  \hline
  広告・販促費 & 40,000 \\
  \hline
  外注費・人件費 & 80,000 \\
  \hline
  その他経費(通信・交通費など) & 20,000 \\
  \hline
  \textbf{営業利益(概算)} & \textbf{100,000} \\
  \hline
\end{tabular}

\vspace{5mm}

%=============================
\section{簡易SWOT分析}
\vspace{2mm}
\begin{tabular}{|p{7cm}|p{7cm}|}
  \hline
  \textbf{強み(Strengths)} & \textbf{弱み(Weaknesses)} \\
  \hline
  経験・スキル/独自の切り口/対応力 & 実績が少ない/資金体力が小さい \\
  \hline
  \textbf{機会(Opportunities)} & \textbf{脅威(Threats)} \\
  \hline
  SNS活用/市場の伸び/時流の変化 & 競合の存在/価格競争/模倣リスク \\
  \hline
\end{tabular}

\vspace{5mm}

%=============================
\section{簡易マインドマップ(事業構想)}

\vspace{1cm}

\begin{center}
\begin{tikzpicture}[
  every node/.style={draw, minimum size=1.1cm, font=\small, align=center},
  node distance=1.5cm and 2.5cm,
  thick
]

% central
\node (core) {事業アイデア};

% branches
\node[above left=of core] (target) {ターゲット\\顧客像};
\node[above right=of core] (value) {提供価値\\差別化};
\node[below left=of core] (model) {収益構造\\価格設定};
\node[below right=of core] (marketing) {集客戦略\\SNSなど};

% connections
\draw[-] (core) -- (target);
\draw[-] (core) -- (value);
\draw[-] (core) -- (model);
\draw[-] (core) -- (marketing);

\end{tikzpicture}
\end{center}

\vspace{5mm}

\section*{※ 補足}
上記の表・図は仮のデータを使った例です。実際の数字や構成はご自身のプランに応じて調整してください。

\end{document}
