\documentclass{article}
\usepackage{tikz}
\usetikzlibrary{intersections}


%円弧の書き方 (aとrはすべて一致させる)
%\draw[thick, blue] (基準点) ++(開始角a:半径)
%arc[start angle=開始角a, end angle=終角b, radius=半径r];

\begin{document}
\begin{tikzpicture}
% 中心点の表示(オプション)
\filldraw[black] (2,2) circle (0.05);
\filldraw[black] (6,6) circle (0.05);

\draw (2,2)--(6,6);

% 円弧の描画
\draw[thick,red] (2,2) ++(-45:3.5) % 開始点(角度-45°、半径3.5)
arc[start angle=-45, end angle=135, radius=3.5];

\draw[thick, blue] (6,6) ++(135:3.5)
arc[start angle=135, end angle=315, radius=3.5];
\end{tikzpicture}


\bigskip

\begin{tikzpicture}
\coordinate (A) at (-3,-3);
\coordinate (B) at (3,3);
\coordinate (P) at (-3,3);

\draw (A)--(B);
\fill[blue] (P) circle (2pt);

%円弧の一部を定義
\draw[blue] (P) ++(0:5)
arc[start angle=0, end angle=-90,radius=5];

%パスの名前を定義
\path[name path=line] (A)--(B);
\path[name path=arc] (P) ++(0:5)
arc[start angle=0, end angle=-90,radius=5]; 

%交点計算
\path[name intersections={of=arc and line, name=int}];

% 交点の描画
\fill[orange] (int-1) circle (2pt) node[below] {$I1$};
\fill[green] (int-2) circle (2pt) node[below] {$I2$};


%作図用の線
\draw[red] (int-1) ++(0:5)
arc[start angle=0, end angle=-25,radius=5];
\draw[red] (int-2) ++(-90:5)
arc[start angle=-90, end angle=-65,radius=5];

%垂線、座標は計算したものを使用した
\draw[blue,thick] (P)--(4,-4);
\end{tikzpicture}
\end{document}
