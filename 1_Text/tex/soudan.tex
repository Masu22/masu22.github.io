\documentclass[12pt]{article}
\usepackage[margin=20mm]{geometry}
\usepackage{titlesec}
\usepackage{enumitem}
\usepackage{amsmath}
\usepackage{graphicx}

\usepackage{enumitem} %itemizeの行間調整のため

\titleformat{\section}{\large\bfseries}{\thesection.}{1em}{}

\setlength{\parindent}{0pt}

\begin{document}

\begin{center}
    {\LARGE \bf 事業相談シート} \\
    \vspace{2mm}
    (中小企業診断士への相談用)
\end{center}

\vspace{1mm}

\section{事業の概要}
\begin{itemize}[leftmargin=1.5em, parsep=4pt]
  %\item 事業名(仮):〇〇サービス
  \item 事業内容:学習塾 (メインは教室での個別指導!オンラインでの解答作成代行)
  \item ターゲット:市内の小中学生がメイン!
  \item 安定してきたら開始したいこと:定期的な講座を開催
  
  高校生対象:難関大学受験の数学物理
  
  大学生・一般対象:大学レベルの専門的な数学
\end{itemize}

\section{事業を始める背景・動機}
(1)\ 塾の経営色の多さ。親身さの欠如\\

(2)\ 中高教育、大学の教育の矛盾点\\
%教育の矛盾点(難関の不定方程式、大学のJordanやetale)\\

(3)\ 実績や経験値\\
%満足率、入塾率、退塾率=送り出し率、8割~、面接(大学推薦、特色)や進学校(地方国立、都内私立)の合格結果
%C,Python.github,aiツール駆使→広告費は抑える

(4) 提供する価値\\
%顧客へのメリット、成績や理解度が上がる!相談場所になる。

(5) 競合との違い(差別化ポイント)\\
%数理特化、超越レベルまで対応、一貫性がある指導

\section{ビジネスモデルの概要}
% 競合の価格平均を求めること: https://jukushiru.com/schools/prefecture/9/city/478
%1カ月 : 10人×最低1万円 = 100000円
\begin{itemize}[leftmargin=1.5em, itemsep=3mm, parsep=5pt]
  \item 営業時間:月~金、14:00~22:00、(土は必要に応じて営業)
  \item 営業形態:1コマ3~5人の個別指導(数理中心)で小中学生がメイン
  
  曜日によって、学年を分ける予定。営業時間外でも、不登校の生徒などの対応を検討している
  
  \item サービスの価格帯(中学生基準、小学生は減額してもよいかも) %1コマ¥3000?
  
  授業料(数学指導)\ 1週間2回(1.5h $\times$ 2回):月額¥10000前後
  
  授業料(数理指導)\ 1週間3回(1.5h $\times$ 3回):月額¥15000前後
  
  土曜講座 (英国のフォロー・テスト対策)\ :1回¥500?
  
  実力・受験対策講座(春夏冬) : 未定
  
  課題や過去問の解答作成(数理、オンラインで受付)\ :問題数で価格を決める?
  
  \item 利益の出し方:月額授業料がメイン、サブとして週末講座や解答作成(不定期)
  \item 経営のコスト:光熱費、プリンタや用紙など
  \item 安定したら開始したいこと:高校生・大学生対象の土曜講座(1回300円?)
  
  難関大学数学対策講座(不定方程式、微積と微積物理、過去問解説など)
  
  大学レベルの数学講座(微積分、ベクトル解析と微分形式、相対性理論、代数幾何学など)
\end{itemize}


\section{現状の課題・相談したいこと}
\begin{itemize}[leftmargin=1.5em]
  \item 市場調査の方法や結果:人口と小中学生の割合、通塾率
  \item 銀行関連の話:自動引き落としの設定方法やコンビニ用紙作成方法
  \item 税金申告の話:会計ソフトfreeeとオンラインでの申告で大丈夫?
  \item 経営戦略や集客戦略について、経営に関して作成・整理する書類はある?
  \item 自宅の2階に看板設置予定。費用や申告の義務について
  %\item 補助金や資金調達について %物件整い次第、出たとこ勝負か!
\end{itemize}



\section{現在の進捗と今後のスケジュール(案)}
\begin{itemize}[leftmargin=1.5em, parsep=4pt]
  \item 現在:準備中(中学生数学教材作成中、物件準備中)
    
  小学、高校は試作品完成。webサイト開発中
  
  \item 3ヶ月後:中学教材完成へ+引っ越し
  \item 来年の3月~4月:本格サービス開始
\end{itemize}

%\section*{備考・その他}
%\begin{itemize}[leftmargin=1.5em]
%  \item 競合例:〇〇社、△△アプリなど
%  \item 顧客像(ペルソナ):29歳女性、都内勤務、SNS利用時間1日2時間
%\end{itemize}

\end{document}
