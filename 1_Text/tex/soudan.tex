\documentclass[12pt]{article}
\usepackage[margin=20mm]{geometry}
\usepackage{titlesec}
\usepackage{enumitem}
\usepackage{amsmath}
\usepackage{graphicx}

\titleformat{\section}{\large\bfseries}{\thesection.}{1em}{}

\setlength{\parindent}{0pt}

\begin{document}

\begin{center}
    {\LARGE \bf 事業相談シート} \\
    \vspace{2mm}
    (中小企業診断士への相談用)
\end{center}

\vspace{3mm}

\section{事業の概要}
\begin{itemize}[leftmargin=1.5em]
  %\item 事業名(仮):〇〇サービス
  \item 事業内容:学習塾 (メインは教室での個別指導!オンラインでの解答作成代行)
  \item ターゲット:市内の小中学生がメイン!
  \item 安定してきたら開始したいこと:定期的な講座を開催
  
  \vspace{0,5mm}
  
  高校生対象:難関大学受験の数学物理
  
  \vspace{0.5mm}
  
  大学生・一般対象:大学レベルの専門的な数学
\end{itemize}

\section{事業を始める背景・動機}
(1)\ 塾の経営色の多さ。親身さの欠如\\

(2)\ 中高教育、大学の教育の矛盾点\\
%教育の矛盾点(難関の不定方程式、大学のJordanやetale)\\

(3)\ 実績や経験値\\
%満足率、入塾率、退塾率=送り出し率、8割~、面接(大学推薦、特色)や進学校(地方国立、都内私立)の合格結果
%C,Python.github,aiツール駆使→広告費は抑える


\section{提供する価値、競合との違い}
\begin{itemize}[leftmargin=1.5em]
  \item 顧客にとってのメリット(例:手間を省く、安価で質が高い)%成績や理解度が上がる!相談場所になる。
  \item 競合との違い(差別化ポイント)%数理特化、超越レベルまで対応、一貫性がある指導
\end{itemize}


\section{ビジネスモデルの概要}
\begin{itemize}[leftmargin=1.5em]
  \item 商品・サービスの価格帯(例:月額2,980円)%週2回で月額10000~15000円
  % 競合の価格平均: https://jukushiru.com/schools/prefecture/9/city/478
  \item 利益の出し方(例:サブスク+初回料金)%月額授業料がメイン、週末講座、解答作成オンライン
  \item コスト構造(例:仕入れ、人件費、広告費)%光熱費、プリンタや用紙
\end{itemize}


\section{現状の課題・相談したいこと}
\begin{itemize}[leftmargin=1.5em]
  \item 市場調査の方法や結果:人口と小中学生の割合、通塾率
  \item 銀行関連の話:自動引き落としの設定方法やコンビニ用紙作成方法
  \item 税金申告の話:会計ソフトfreeeとオンラインでの申告で大丈夫?
  \item 経営戦略や集客戦略について、経営に関して作成・整理する書類はある?
  \item 自宅の2階に看板設置予定。費用や申告の義務について
  %\item 補助金や資金調達について %物件整い次第、出たとこ勝負か!
\end{itemize}



\section{現在の進捗と今後のスケジュール(案)}
\begin{itemize}[leftmargin=1.5em]
  \item 現在:準備中(試作品あり、SNSアカウント開設済) %教材つくらねば!!まず中学+小学少しづつ。動画はまだ先かな。。。
  \item 3ヶ月後:テスト販売 %引越ししつつ、宿題代行もちょいちょい
  \item 半年後:本格サービス開始
\end{itemize}

%\vspace{5mm}

%\section*{備考・その他}
%\begin{itemize}[leftmargin=1.5em]
%  \item 競合例:〇〇社、△△アプリなど
%  \item 顧客像(ペルソナ):29歳女性、都内勤務、SNS利用時間1日2時間
%\end{itemize}

\end{document}
