\documentclass[12pt]{article}
\usepackage[margin=20mm]{geometry}
\usepackage{titlesec}
\usepackage{enumitem}
\usepackage{amsmath}
\usepackage{graphicx}
\usepackage{tikz}

\usepackage[normalem]{ulem} %下線のカスタマイズ
\usepackage{enumitem} %itemizeの行間調整のため

\titleformat{\section}{\large\bfseries}{\thesection.}{1em}{}

%自動字下げをなくす
\setlength{\parindent}{0pt}

%下線の太さ変更
\renewcommand{\ULthickness}{1.5pt}

\begin{document}

\begin{center}
    {\LARGE \bf 事業相談シート} \\
\end{center}

\vspace{1mm}

\section{事業の概要}
\begin{itemize}[leftmargin=1.5em, itemsep=3mm, parsep=4pt]
  %\item 事業名(仮):〇〇サービス
  \item 事業内容:学習塾 (メインは教室での個別指導!) + サブとして、解答作成なども
  \item ターゲット:市内の小中学生がメイン!
  \item 安定してきたら開始したいこと:定期的な講座を開催
  
  高校生対象:難関大学受験の数学物理
  
  大学生・一般対象:大学レベルの専門的な数学
\end{itemize}

\section{事業を始める背景・動機}
\begin{itemize}[leftmargin=1.5em, itemsep=3mm, parsep=4pt]
\item \uline{\textbf{一人一人に寄り添える塾、アットホームな塾を目指して}}

生徒一人一人と対話して、生徒に寄り添える塾を目指します。

一人の講師が一貫して、全ての生徒に向き合います。

収益を優先する学習塾も多いですが、私は学習面・精神面の両面で支えられる、そんなアットホームな塾を目指しています。

成績や理解度が上がだけではなく、相談場所にもなれる、そんな塾でありたい。


学習内容の質問に限らず、入試に対する不安、進路相談、面接対策など、どんなことでも相談に乗ります。

また、単に成績が上がるだけでなく、「本質的にわかる」「再現できる」という学習体験を重視しています。

\item \uline{\textbf{教育の問題点、受験と学校教育の落差(とくに、数理にこの傾向が強い)}}

授業が分かりにくい、休んだ人に対する補講がない、教科書に要点がまとまっていないなど学校教育の問題点がある。

高校・大学受験に必要な知識の不足が目立つ。学校の授業内容だけでは、受験に太刀打ちできない。現状の教育だと、過去問を解く際に大切になる、再現性のある解き方が身に付きにくい。

(高校受験の関数問題や規則性問題、難関大学の不定方程式系の問題や積分などなど)

\item \uline{\textbf{大学の数学教育において、生徒自身の独学が多く、最先端理論までたどり着きにくい。}}

時間の都合上、分野をまたがる話や最先端の内容につながる丁寧な説明が不足しており、授業では解説を行わないことが多いです。

そのため、いろいろな学習資料を、独学で学んでいくことになります。そこでは、アイデアの発想やつながらが見えにくいことが多く大変です。

(Jordan標準形、スキームとetale射の幾何、接続と微分幾何など)

そこで、基礎~専門基礎につながる部分を整備する必要があります。

これをオンラインのテキストや講座などで補っていきたい。

\newpage

\item \uline{\textbf{実績や経験値}}

直近の6年間の指導実績として、入塾率、満足率はいずれも8割以上を誇り、進学校(地方国立大学、中学受験、県内トップレベルの進学高校)への合格者も多数輩出しています。

また、高校受験の特色入試、大学推薦入試(面接・特色入試)の合格者も出ております。

今後も、より多くの生徒の受験合格をサポートしていきたいと思います。

\item \uline{\textbf{競合との違い(差別化ポイント)}}

経験上、数理が苦手となる生徒は多い。

数理に特化して、専門的なのものも含めて、いろいろなレベルを対応できる強みがあると思う。

また、一人の講師が一貫性をもって、生徒を見ることができる。

要点やポイントをまとめた独自教材を使用していくことで、より深い理解を養うことができるはず。
\end{itemize}

以上のような観点から、新しく事業を開始したいと思いました。

\section{ビジネスモデルの概要}
% 競合の価格平均を求めること: https://jukushiru.com/schools/prefecture/9/city/478
%コマ数 : 1日5コマ × 平日5日 = 25コマ
%max人数 : 全枠 (25コマ×4人) ÷ 3枠/1人 = 33 人 
%1カ月max目安 : 30人×12000円 = 360000円
\begin{itemize}[leftmargin=1.5em, itemsep=4mm, parsep=5pt]
  \item 営業時間:月~金、14:00~22:00、(土は必要に応じて営業)
  \item 営業形態:1コマ3~5人の個別指導(数理中心)で小中学生がメイン
  
  曜日によって、学年を分ける予定。営業時間外でも、不登校の生徒などの対応を検討している
  
  \item サービスの価格帯(中学生基準、小学生は減額してもよいかも) %1コマ¥3000?
  
  授業料(数学指導)\ 1週間2回(1.5h $\times$ 2回):月額¥10000前後
  
  授業料(数理指導)\ 1週間3回(1.5h $\times$ 3回):月額¥15000前後
  
  土曜講座 (英国のフォロー・テスト対策)\ :1回¥500?
  
  実力・受験対策講座(春夏冬) : 未定
  
  課題や過去問の解答作成(数理、オンラインで受付)\ :問題数で価格を決める?
  
  \item 利益の出し方:月額授業料がメイン、サブとして週末講座や解答作成(不定期)
  \item 経営のコスト:光熱費、プリンタや用紙など %C,Python.github,aiツール駆使→広告費は抑える
  \item 安定したら開始したいこと:高校生・大学生対象の土曜講座(1回300円?)
  
  難関大学数学対策講座(不定方程式、微積と微積物理、過去問解説など)
  
  大学レベルの数学講座(微積分、ベクトル解析と微分形式、相対性理論、代数幾何学など)
\end{itemize}

\newpage

\section{現状の課題・相談したいこと}
\begin{itemize}[leftmargin=1.5em]
  \item 市場調査の方法や結果:人口と小中学生の割合、通塾率
  \item 銀行関連の話:代金の受領方法について。\\
  事業専用の銀行口座を作れる?自動引き落としの設定方法やコンビニ用紙作成方法
  \item 引き落としなどは、先払いと後払いのどちらがよい?返金対応はどうするればよい?
  \item 税金申告の話:会計ソフトfreeeとオンラインでの申告で大丈夫?
  \item 経営戦略や集客戦略について、経営に関して作成・整理する書類はある?
  \item 自宅の2階に看板設置予定。費用や申告の義務について
  %\item 補助金や資金調達について %物件整い次第、出たとこ勝負か!
\end{itemize}


\section{現在の進捗と今後のスケジュール(案)}
\begin{itemize}[leftmargin=1.5em, parsep=4pt]
  \item 現在:準備中(中学生数学教材作成中、物件準備中)
    
  小学、高校は試作品完成。webサイト開発中
  
  \item 3ヶ月後:中学教材完成へ+引っ越し完了
  \item 今年中:小学教材(5,6年)完成へ + できれば講座テキストの試作品「微積分」へ着手
  
  \vspace{4mm} \hrule height 1.5pt \vspace{3mm}
  
  \item 来年の3月~4月:本格サービス開始
\end{itemize}

%\section*{備考・その他}
%\begin{itemize}[leftmargin=1.5em]
%  \item 競合例:〇〇社、△△アプリなど
%  \item 顧客像(ペルソナ):29歳女性、都内勤務、SNS利用時間1日2時間
%\end{itemize}

\end{document}
