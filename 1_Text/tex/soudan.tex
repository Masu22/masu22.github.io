\documentclass[a4paper,12pt]{article}
\usepackage[margin=25mm]{geometry}
\usepackage{titlesec}
\usepackage{enumitem}
\usepackage{amsmath}
\usepackage{graphicx}

\titleformat{\section}{\large\bfseries}{\thesection.}{1em}{}

\begin{document}

\begin{center}
    {\LARGE \bf 事業相談シート} \\
    \vspace{2mm}
    (中小企業診断士への相談用)
\end{center}

\vspace{5mm}

\section{事業の概要}
\begin{itemize}[leftmargin=1.5em]
  \item 事業名(仮):〇〇サービス
  \item 内容:〇〇を活用して△△を提供するサービス
  \item ターゲット:20代〜30代の女性(都市部在住、SNS利用者)
\end{itemize}

\section{事業を始める背景・動機}
% きっかけ:塾の親身さ欠如、教育の矛盾点(難関の不定方程式、大学のJordanやetale)
\begin{itemize}[leftmargin=1.5em]
  \item きっかけとなった経験や課題意識 %満足率、入塾率、退塾率=送り出し率、8割~、面接(大学推薦、特色)や進学校(地方国立、都内私立)の合格結果
  \item 自分の強みやスキル(例:デザイン、プログラミング) %C,Python.github,aiツール駆使→広告費は抑える
\end{itemize}

\section{提供する価値(バリュープロポジション)}
\begin{itemize}[leftmargin=1.5em]
  \item 顧客にとってのメリット(例:手間を省く、安価で質が高い)
  \item 競合との違い(差別化ポイント)
\end{itemize}

\section{ビジネスモデルの概要}
\begin{itemize}[leftmargin=1.5em]
  \item 商品・サービスの価格帯(例:月額2,980円)
  \item 利益の出し方(例:サブスク+初回料金)
  \item コスト構造(例:仕入れ、人件費、広告費)
\end{itemize}

\section{現状の課題・相談したいこと}
% 市場調査:人口と小中学生の割合、通塾率
% 競合の価格平均: https://jukushiru.com/schools/prefecture/9/city/478
% 銀行関連:自動引き落としやコンビニ用紙作成
% 税金報告:freeeでやれるか??
% 経営戦略や経営書類は?
% 看板の設置(自宅)の費用や申告は??
\begin{itemize}[leftmargin=1.5em]
  \item 市場調査の方法
  \item 集客戦略
  \item 補助金や資金調達について %物件整い次第、出たとこ勝負か!
\end{itemize}

\section{現在の進捗と今後のスケジュール(案)}
\begin{itemize}[leftmargin=1.5em]
  \item 現在:準備中(試作品あり、SNSアカウント開設済) %教材つくらねば!!まず中学+小学少しづつ。動画はまだ先かな。。。
  \item 3ヶ月後:テスト販売 %引越ししつつ、宿題代行もちょいちょい
  \item 半年後:本格サービス開始
\end{itemize}

\section*{備考・その他}
\begin{itemize}[leftmargin=1.5em]
  \item 競合例:〇〇社、△△アプリなど
  \item 顧客像(ペルソナ):29歳女性、都内勤務、SNS利用時間1日2時間
\end{itemize}

\end{document}
