\documentclass[tikz,border=10pt]{article}
\usepackage{tikz}
\usetikzlibrary{math}
\usepackage{tikz-3dplot}
\usepackage{luatexja}
\usepackage{pgfplots}

\pgfplotsset{compat=1.18} % ← 警告対策!
\usetikzlibrary{intersections} %交点計算用

\begin{document}

\tdplotsetmaincoords{75}{128} 

\begin{tikzpicture}[tdplot_main_coords]

% xy平面を青で塗る
\draw[black,fill=gray!50, opacity=0.5] (0,0,0) -- (6,0,0) -- (6,6,0) -- (0,6,0) -- cycle;

% yz平面を赤で塗る
\draw[black,fill=gray!50, opacity=0.5] (0,0,0) -- (0,6,0) -- (0,6,6) -- (0,0,6) --cycle;

% 3D軸
%\draw[thick,->] (0,0,0) -- (7,0,0) node[below] {$x$}; % x軸
%\draw[thick,->] (0,0,0) -- (0,7,0) node[right] {$y$}; % y軸
%\draw[thick,->] (0,0,0) -- (0,0,7) node[above] {$z$}; % z軸

%四角錐の平面図
\draw[red,fill=red!20] (2,2,0) -- (2,4,0) -- (4,4,0) -- (4,2,0) -- cycle;

%立面図
\draw[blue,fill=blue!20] (0,2,2) -- (0,4,2) -- (0,3,5) -- cycle;

%実体の四角錐
\draw[thick,fill=black!40] (3,3,5)--(2,4,2)--(4,4,2)--cycle;
\draw[thick,fill=black!40] (3,3,5)--(4,2,2)--(4,4,2)--cycle;
\draw[thick,dashed] (4,2,2)--(2,2,2)--(2,4,2);
\draw[thick] (4,2,2)--(4,4,2)--(2,4,2);
\draw[thick,dashed] (3,3,5)--(2,2,2);

%平面図への投影
\draw[dashed,red] (2,2,2)--(2,2,0);
\draw[dashed,red] (2,4,2)--(2,4,0);
\draw[dashed,red] (4,2,2)--(4,2,0);
\draw[dashed,red] (4,4,2)--(4,4,0);

%立面図への投影
\draw[dashed,blue] (3,3,5)--(0,3,5);
\draw[dashed,blue] (2,2,2)--(0,2,2);
\draw[dashed,blue] (2,4,2)--(0,4,2);
\end{tikzpicture}


\bigskip

%円弧と移動
\begin{tikzpicture}
%原点からスタートした円弧を書く!
\filldraw (0,0) circle(2pt) node[below right] {$O$};
%(0,0)をスタートとした円弧
\draw[gray, dashed] (0,0) arc[start angle=0, end angle=180, radius=2];


%移動後の原点P
\coordinate (P) at (30:2);

%線分 y=(tan30)x
\draw (210:5)--(30:5);
\draw[dashed] (-5,0)--(5,0);

%20度ずらして、30度の平行移動

%平行移動+回転してy=(tan30)xに合わせた円弧
\begin{scope}[shift={(P)}]  %平行移動
 \begin{scope}[rotate around={30:(0,0)}] %30度回転
  \draw[blue] (0,0) arc[start angle=0, end angle=180, radius=2];
  \end{scope}
\end{scope}
\end{tikzpicture}

\bigskip


%垂線の作図
\begin{tikzpicture}[scale=1]
% 軸
%\draw[->] (-1,0) -- (4,0) node[right] {$x$};
%\draw[->] (0,-1) -- (0,5) node[above] {$y$};

%元の直線
\draw[domain=-3:3,smooth,variable=\x,thick, name path=A] plot ({\x},{(-0.5)*\x}) node[right] {元の直線};

%点P
\coordinate (P) at (2,3);
\filldraw[red] (P) circle (2pt) node[right] {$P$};

%P中心の円
\draw[red, domain=210:280, samples=100, smooth, variable=\t, name path=B] plot ({2 + 4*cos(\t)}, {3 + 4*sin(\t)});

%交点の取得
\path [name intersections={of=A and B, name=I}];
\fill[red] (I-1) circle (2pt);
\fill[red] (I-2) circle (2pt);

%交点のx,y座標を数値として取得
%参考:https://note.com/kou_no_note/n/n4cf031e481a1
\tikzmath{
coordinate \c,\d;
coordinate \cbase; %ptを座標数値に変換するためのもの
real \x,\y,\u,\v;
%座標の定義
\cbase=(1,1);
\c=(I-1);
\d=(I-2);
%ptをなくすために計算
\x = \cx/\cbasex;
\y = \cy/\cbasey;
\u = \dx/\cbasex;
\v = \dy/\cbasey;
}

%交点中心の円たち(交点はpt座標なので、円の半径は長さ(cm)で指定した)
\draw[red, domain=260:280, samples=100, smooth, variable=\t, name path=C] plot ({\x + 4*cos(\t)}, {\y + 4*sin(\t)});
\draw[cyan, domain=210:225, samples=100, smooth, variable=\t, name path=D] plot ({\u + 4*cos(\t)}, {\v + 4*sin(\t)});

%交点の取得
\path [name intersections={of=C and D, name=J}];
\fill[blue] (J-1) circle (2pt);

%垂線
\draw[thick,blue] ($(P)!1.2!(J-1)$)--($(P)!-0.2!(J-1)$) node[above] {垂線};
\end{tikzpicture}


\bigskip


%垂直二等分線
\begin{tikzpicture}
\coordinate (A) at (-3,0);
\coordinate (B) at (3,0);

\node[above] at (A) {$A$};
\node[above] at (B) {$B$};

\fill (A) circle (2pt);
\fill (B) circle (2pt);

%直線
\draw[thick] (-5,0)--(5,0);

%作図の円
\draw[red,domain=-70:70,samples=100,smooth,variable=\t,name path=C1] plot ({-3+4*cos(\t)}, {4*sin(\t)});
\draw[red,domain=110:250,samples=100,smooth,variable=\t,name path=C2] plot ({3+4*cos(\t)}, {4*sin(\t)});

%交点
\path [name intersections={of=C1 and C2, name=Inter}];
\fill[red] (Inter-1) circle (2pt);
\fill[red] (Inter-2) circle (2pt);

%垂直二等分線
\draw[blue,thick] ($(Inter-1)+(0,-2)$)--($(Inter-2)+(0,2)$);
\end{tikzpicture}


\end{document}
