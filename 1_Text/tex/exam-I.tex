% lualatexでコンパイル
% 大試、院試の解答作成

\documentclass[11pt]{article}

\usepackage{makecell} %tabularの横線の太さ調整のため
\usepackage[table]{xcolor} %表に色をつけるため
\usepackage[most]{tcolorbox}
\usepackage{luatexja}
\usepackage{amsmath,amssymb,amsfonts}
\usepackage[margin=3cm]{geometry}


\setlength{\parindent}{0pt}
\linespread{1.2} %行間を調整

%tcolorboxのスタイルを設定
\tcbset{
  mybox/.style={
    %colback=yellow!5,
    %colframe=orange!80!black,
    fonttitle=\bfseries,
    title=#1
  }
}

\begin{document}
\begin{center}
{\LARGE \textbf{京都大学院 	??年 専門}}
\end{center}
\vspace{2mm}

\section*{各問題の内容と感想}

\renewcommand{\arraystretch}{1.3} %表の行間調整
\begin{tabular}{!{\vrule width 1pt}c!{\vrule width 1pt}c!{\vrule width 1pt}c!{\vrule width 1pt}c!{\vrule width 1pt}}
\Xhline{1pt}
\rowcolor{gray!20} 問題番号 & 内容 & 感想 & なかけん難易度\\
\Xhline{1pt} % 太さ 倍の横線
1 & ?? & ?? & \\
\Xhline{1pt}
2 & ?? & ??&★★★☆☆\\
\Xhline{1pt}
3 & ?? & ?? & \\
\Xhline{1pt}
4 & ?? & ?? & \\
\Xhline{1pt}
5 & ?? & ?? & \\
\Xhline{1pt}
\end{tabular}

\vspace{1cm}


\vfill
\hrule
\vspace{1mm}
{\footnotesize ここは、脚注の文}

\newpage


\begin{tcolorbox}[mybox={問?}]
有限体$\mathbb{F}_p$上の多項式環を$\mathbb{F}_p [X]$として、その部分環を$A$とする。\\
$\mathbb{F}_p$ベクトル空間の次元が$\text{dim}_{\mathbb{F}_p} (\mathbb{F}_p / A) =1$となる$A$をすべて求められるか??
\end{tcolorbox}

\underline{\textbf{考え方・疑問点}}\\
$\bullet$ \\

\underline{\textbf{解き方}}\\


\end{document}