% lualatexでコンパイル
% 1年中学数学目次下書き

\documentclass[11pt]{article}

\usepackage[most]{tcolorbox}
\usepackage{luatexja}

\setlength{\parindent}{0pt}
\linespread{1.2} %行間を調整

%tcolorboxのスタイルを設定
\tcbset{
  mybox/.style={
    %colback=yellow!5,
    %colframe=orange!80!black,
    fonttitle=\bfseries,
    title=#1
  }
}

\begin{document}
\section{文字式の導入}
\subsection*{文字とは?}
数学では「分からない数」や「変わる数」を使うことが多い\\
そんな数を$a$や$x$などの\textbf{文字}を使って表す!

\subsection*{どうして文字を使うの?}
\fbox{例1}\\
チョコ1個120円で、ガムは1個80円です。\\
チョコを1個、ガムを3個買うと、合計で何円になるでしょうか?\\
$120 \times 1 + 80 \times 3 = 120 + 240 =360$なので、合計で360円になる。\\

\fbox{例2}\\
チョコを1個、ガムを何個か買いました。ガムの個数が分からないけど、何とか式であらわしたい\\
ガムの個数が分からないので、ガムの個数を$a$としてみる。\\
$a$個ガムを買ったとすると、合計金額は、$120 \times 1 + 80 \times a =120 + 80 \times a$ (円)となる。\\
こんな感じに、文字を使うと、式が作れるようになるから便利だね!\\

あとでガムを5個買ったわかったとき、金額を計算することもできる!\\
ガムの個数を$a$としていたから、$a$を5にすると、合計金額は\\
$120 \times 1 + 80 \times a = 120 \times 1 + 80 \times 5 =120 + 400 = 520$円となる。\\
こんな感じで、式を作っておくと、$a$を変えるだけで、すぐに合計金額が計算できる!\\

\begin{tcolorbox}[mybox={文字のメリット}]
\begin{itemize}
\item 分からないものを文字にして、式がつくれる
\item 文字に数字を入れるだけで、計算できる
\end{itemize}
\end{tcolorbox}

文字を使った式には、書き方や計算のルールがあるので、それを今後勉強していこう!\\

\begin{tcolorbox}[mybox={文字式の書き方のルール}]
\begin{itemize}
\item $\times$は省く
\item 数字はアルファベットの前に書く
\item 文字はアルファベット順にならべる
\item 文字の前の1は省く
\item $\pi$などのギリシャ文字は、数字とアルファベットの間にかく
\end{itemize}
\end{tcolorbox}

\fbox{例}

\begin{tabular}{llr}
(1) & $2 \times a = 2a$ & \\
(2) & $b \times a = ab$ &\\
(3) & $1 \times x = 1x =x$ & \hspace{2cm} $y \times (-1) = -1y =-y$ \\
(4) & $3 \times r \times \pi = 3\pi r$ &\\
\end{tabular}


\begin{tcolorbox}[mybox={文字式の積}]
文字式同士のかけ算のやり方をまとめる。
\begin{itemize}
\item 文字式の前の数字をかける
\item 文字はアルファベット順にならべる
\item 同じ文字は指数を使って表す
\end{itemize}
\end{tcolorbox}

\fbox{例}

\begin{tabular}{ll}
(1) & $2b \times 3a = 2 \times 3 \ b \ a = 6 ab$\\
(2) & $b \times 3b = 1 \times 3\ b \  b = 3b^2$\\
\end{tabular}

\begin{tcolorbox}[mybox={文字式の商}]
\begin{itemize}
\item 数字をわる(分数なら逆数かけ)、後ろの文字は分母に
\item 分数にして、数字の部分を約分する
\item 分母と分子に同じ文字があったら、なくす
\item 分数を整理して、完成 (マイナス符号、分母の1に注意)
\end{itemize}
\end{tcolorbox}

\fbox{例}

\begin{tabular}{ll}
(1) & $2b \times 3a = 2 \times 3 \ b \ a = 6 ab$\\
(2) & $b \times 3b = 1 \times 3\ b \  b = 3b^2$\\
\end{tabular}


\section{文字式と単位、数量の表し方}
\begin{itemize}
\item 文字と単位を区別したい
\item 単位には、(\ )をつける
\item \fbox{例} \ \ $a$\ (kg)\ ,\ $d$\ (m)
\item 代金、速さ、平均
\item 整数の表し方、偶数と奇数
\end{itemize}


\section{文字への代入、式の値}
\begin{itemize}
\item 文字と数字の間には、$\times$がある。
\item 文字のところを数字にかえて、計算
\item \fbox{例} \ \ $2a+1 = 2 \times a +1 = 2 \times 5 +1 =11$
\item 負の数は、(\ )をつけて代入する
\end{itemize}

\section{文字式の計算、式の加減}
\begin{itemize}
\item 加減:(\ )をはずして、同じ文字の前の数字をたす
\item 3つの$\div$や$\times$は、順番にやろう!
\item 分配法則:数字と(\ )が並ぶ\\ 
$3(2a+4)=3 \times 2a + 3 \times 4 =6a+12$
\end{itemize}

\section{関係を表す式}
\begin{itemize}
\item 等式、不等式で表す
\item 以上、以下、未満、より大きいを記号で表す
\end{itemize}

\newpage

{\LARGE \noindent 方程式}

\section{方程式の性質}
\begin{itemize}
\item 両辺に加減乗除してもよい
\item 左右をいれかてもよい
\item 移項
\end{itemize}

\section{方程式の解き方}
\begin{itemize}
\item 方程式の性質をつかって、$x=○○$にする
\item $x$に答えを代入して、左右が等しくなるか?確かめる
\item 分数や小数は、何倍かしてなくしてから計算する
\end{itemize}

\section{比例式}
内側と外側をかける

\section{方程式の利用}
\begin{itemize}
\item 分からない数量を文字$x$でおく。
\item 数量を整理して、方程式を立てる。
\item 方程式を解く。
\item 解が方程式の答えになっているか?数量にあっているか?確認する。
\end{itemize}

\newpage

{\LARGE \noindent 平面図形}

\section{直線、角、移動}
\begin{itemize}
\item 直線、線分、半直線、角
\item 垂直と平行
\item 図形の移動(平行、回転、線対称、点対称)
\end{itemize}

\section{基本の作図}
\begin{itemize}
\item 垂直二等分線
\item 角の二等分線
\item 垂線
\end{itemize}

\section{円}
\begin{itemize}
\item 円の中心と半径、直径
\item 弧、弦、おうぎ形
\item 円の接線
\item 円周と面積の公式
\item おうぎ形の弧の長さ$\ell$と面積の公式$S$、$S=\frac{1}{2} \ell r$
\end{itemize}

\newpage

{\LARGE \noindent 空間図形}
\section{空間図形}
\begin{itemize}
\item 角錐と円錐
\item 展開図
\item 正多面体
\item 回転体
\end{itemize}

\section{直線と平面の位置関係}
\begin{itemize}
\item 2平面(交わる、平行)
\item 直線と平面(交わる、平行、平面上)
\item 2直線(交わる、平行、ねじれ)
\end{itemize}

\section{投影図}

\section{表面積、体積}
\begin{itemize}
\item 角柱、円柱の表面積、体積
\item 角錐、円錐の表面積、体積
\item 球の表面積、体積
\end{itemize}

\newpage

{\LARGE \noindent データの活用}
\section{度数分布表とヒストグラム}
\begin{itemize}
\item 度数分布表、累積度数、相対度数、累積相対度数
\item ヒストグラムと度数折れ線
\item 平均値、範囲、中央値(メジアン)、最頻値(モード)
\item 確率とは、、、
\end{itemize}

\end{document}
