\documentclass{jsarticle}

\begin{document}
{\LARGE \noindent 文字式}

\section{文字式の導入}


\subsection*{文字とは?}
数学では「**分からない数**」や「**変わる数**」を使うことが多くなるよ!\\

そんなときに使うのが…\\
👉 **文字**(x, a, b など)!


\subsection*{どうして文字を使うの?}

例1\\
* チョコ1個120円で、ガムは1個80円\\
* チョコを1個、ガムを3個か買いました\\
* 合計で何円になるでしょうか??\\

$120 \times 1 + 80 \times 3 = 120 + 240 =360$\\
なので、合計で360円になる。\\

例2\\
* チョコ1個120円で、ガムは1個80円\\
* チョコを1個、ガムを何個か買いました\\
* 合計で何円になるでしょうか??\\

ガムの個数が分からないので、文字$a$を使ってみる。\\
$a$個ガムを買ったとすると、合計金額は、$120 \times 1 + 80 \times a =120 + 80 \times a$ (円)となる。\\
**文字**を使うと、式が作れるようになるから便利だね!\\

それから、ガムが5個買ったわかったとき、$a$を5にすると、合計金額が計算できる。\\
$120 \times 1 + 80 \times 5 =120 + 400 = 520$円となる。\\

こんな感じで、式を作っておくと、$a$を変えるだけで、すぐに合計金額が計算できる!\\


* 文字を使った式には、書き方のルールがあるので、それを今後勉強していこう!\\


\newpage

\section{文字式の書き方のルール}
\begin{itemize}
\item 文字の前に数字を
\item 文字はアルファベット順
\item 文字の前の1は省略
\end{itemize}

\section{文字式の積}
\begin{itemize}
\item 文字の前の数字をかける
\item 文字はアルファベット順にならべる
\item 同じ文字は指数を使って表す
\end{itemize}

\section{文字式の商}
\begin{itemize}
\item 分数にして、数字の部分を約分する
\item 分母と分子に同じ文字があったら、なくす
\item 分数を整理して、完成 (マイナス符号、分母の1に注意)
\end{itemize}

\section{文字への代入}
\begin{itemize}
\item 文字と数字の間には、$\times$がある。
\item 文字のところを数字にかえて、計算
\item \fbox{例} \ \ $2a+1 = 2 \times a +1 = 2 \times 5 +1 =11$
\end{itemize}

\section{文字式の計算}
\begin{itemize}
\item $+$や$-$は省略できない
\item 3つの$\div$や$\times$は、順番にやろう!
\end{itemize}

\section{文字式と単位}
\begin{itemize}
\item 文字と単位を区別したい
\item 単位には、(\ )をつける
\item \fbox{例} \ \ $a$\ (kg)\ ,\ $d$\ (m)
\end{itemize}


\newpage

{\LARGE \noindent 方程式}

\section{方程式の性質}
\begin{itemize}
\item 両辺に加減乗除してもよい
\item 左右をいれかてもよい
\end{itemize}

\section{方程式の解き方}
\begin{itemize}
\item 方程式の性質をつかって、$x=○○$にする
\item $x$に答えを代入して、左右が等しくなるか?確かめる
\item 分数や小数は、何倍かしてなくしてから計算する
\end{itemize}

\end{document}
