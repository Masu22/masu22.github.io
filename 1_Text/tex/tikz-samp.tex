\documentclass[tikz,border=10pt]{article}
\usepackage{tikz}
\usepackage{tikz-3dplot}

\begin{document}

\tdplotsetmaincoords{75}{128} 

\begin{tikzpicture}[tdplot_main_coords]

% xy平面を青で塗る
\draw[black,fill=gray!50, opacity=0.5] (0,0,0) -- (6,0,0) -- (6,6,0) -- (0,6,0) -- cycle;

% yz平面を赤で塗る
\draw[black,fill=gray!50, opacity=0.5] (0,0,0) -- (0,6,0) -- (0,6,6) -- (0,0,6) --cycle;

% 3D軸
%\draw[thick,->] (0,0,0) -- (7,0,0) node[below] {$x$}; % x軸
%\draw[thick,->] (0,0,0) -- (0,7,0) node[right] {$y$}; % y軸
%\draw[thick,->] (0,0,0) -- (0,0,7) node[above] {$z$}; % z軸

%四角錐の平面図
\draw[red,fill=red!20] (2,2,0) -- (2,4,0) -- (4,4,0) -- (4,2,0) -- cycle;

%立面図
\draw[blue,fill=blue!20] (0,2,2) -- (0,4,2) -- (0,3,5) -- cycle;

%実体の四角錐
\draw[thick,fill=black!40] (3,3,5)--(2,4,2)--(4,4,2)--cycle;
\draw[thick,fill=black!40] (3,3,5)--(4,2,2)--(4,4,2)--cycle;
\draw[thick,dashed] (4,2,2)--(2,2,2)--(2,4,2);
\draw[thick] (4,2,2)--(4,4,2)--(2,4,2);
\draw[thick,dashed] (3,3,5)--(2,2,2);

%平面図への投影
\draw[dashed,red] (2,2,2)--(2,2,0);
\draw[dashed,red] (2,4,2)--(2,4,0);
\draw[dashed,red] (4,2,2)--(4,2,0);
\draw[dashed,red] (4,4,2)--(4,4,0);

%立面図への投影
\draw[dashed,blue] (3,3,5)--(0,3,5);
\draw[dashed,blue] (2,2,2)--(0,2,2);
\draw[dashed,blue] (2,4,2)--(0,4,2);
\end{tikzpicture}


\bigskip


\begin{tikzpicture}
%原点からスタートした円弧を書く!
\filldraw (0,0) circle(2pt) node[below right] {$O$};
%(0,0)をスタートとした円弧
\draw[gray, dashed] (0,0) arc[start angle=0, end angle=180, radius=2];


%移動後の原点P、回転角度は30度
\coordinate (P) at (30:2);
%10度だけ下からスタートするために修正
\coordinate (Q) at (20:2);


%線分 y=(tan30)x
\draw (210:5)--(30:5);
\draw[dashed] (-5,0)--(5,0);


%平行移動+回転してy=(tan30)xに合わせた円弧
\begin{scope}[shift={(P)}]  %平行移動
 \begin{scope}[rotate around={30:(0,0)}] %30度回転
  \draw[blue] (0,0) arc[start angle=0, end angle=180, radius=2];
  \end{scope}
\end{scope}

\begin{scope}[shift={(Q)}]  %平行移動
 \begin{scope}[rotate around={30:(0,0)}] %30度回転
  \draw[red] (0,0) arc[start angle=0, end angle=180, radius=2];
  \end{scope}
\end{scope}

\end{tikzpicture}
\end{document}
