\documentclass{jsarticle}

\begin{document}

%文字式の導入 ~分からない数を文字であらわそう

\section{文字式の導入}


\subsection*{文字とは?}
数学では「**分からない数**」や「**変わる数**」を使うことが多くなるよ!\\

そんなときに使うのが…\\
👉 **文字**(x, a, b など)!


\subsection*{どうして文字を使うの?}

例1\\
* チョコ1個120円で、ガムは1個80円\\
* チョコを1個、ガムを3個か買いました\\
* 合計で何円になるでしょうか??\\

$120 \times 1 + 80 \times 3 = 120 + 240 =360$\\
なので、合計で360円になる。\\

例2\\
* チョコ1個120円で、ガムは1個80円\\
* チョコを1個、ガムを何個か買いました\\
* 合計で何円になるでしょうか??\\

ガムの個数が分からないので、文字$a$を使ってみる。\\
$a$個ガムを買ったとすると、合計金額は、$120 \times 1 + 80 \times a =120 + 80 \times a$ (円)となる。\\
**文字**を使うと、式が作れるようになるから便利だね!\\

それから、ガムが5個買ったわかったとき、$a$を5にすると、合計金額が計算できる。\\
$120 \times 1 + 80 \times 5 =120 + 400 = 520$円となる。\\

こんな感じで、式を作っておくと、$a$を変えるだけで、すぐに合計金額が計算できる!\\


* 文字を使った式には、書き方のルールがあるので、それを今後勉強していこう!\\


\begin{itemize}
\item 文字式の積のルール
\item 文字式の商のルール、逆数
\item $+-$は省略できない、単位の扱い、文字への代入
\item 文字式の計算例
\item 方程式と基本解法、確かめ算
\end{itemize}

---
---

はい、まさにその順番は**非常に理にかなっていますし、教育的にもベストな流れ**です!
以下のような順序で指導・整理するのは、**混乱を防ぎ、一般化も促しやすい**です。

## 🔷 文字式の商:処理のおすすめ手順

---

### ✅ Step ①:**符号を先に決める**

* まず、「マイナスが何個あるか」で符号を決定!

| 分子・分母の符号   | 答えの符号 |
| ---------- | ----- |
| $+ \div +$ | $+$   |
| $+ \div -$ | $-$   |
| $- \div +$ | $-$   |
| $- \div -$ | $+$   |

💡これを「符号だけ先に計算する」と整理することで、**後の文字や指数の処理に集中できる**ようになります。

---

### ✅ Step ②:**分数(商)にして整理**

* 分数の形にして、**係数と文字をそれぞれ約分**する

#### 例:

$$
\frac{-12x^3}{4x} = - \frac{12x^3}{4x} = -3x^2
$$

* ここで、分子・分母の文字は指数法則で整理
* 係数は整数の割り算(普通の約分)

---

### ✅ Step ③:または逆数を使ってかけ算として整理

これはやや発展ですが、分数が苦手な生徒にも有効な表現です。

$$
\frac{a}{b} = a \cdot \frac{1}{b}
$$

なので、

$$
\frac{-12x^3}{4x} = -12x^3 \cdot \frac{1}{4x} = -3x^2
$$

---

## 🔷 この順番が効果的な理由

| 理由              | 解説                            |
| --------------- | ----------------------------- |
| ✅ 符号で混乱しなくなる    | 最初に符号だけ処理すると、後の作業が「純粋な式変形」になる |
| ✅ 約分・指数計算に集中できる | 計算ミスが減り、理解も深まる                |
| ✅ 発展的な構造に接続できる  | 逆数・等式変形・関数など、高校内容とも接続しやすい     |

---

## 🔷 実際の説明モデル(授業例)

> 💬「まず、符号だけ考えよう! マイナス1個 → マイナス。次に、式全体を分数の形にして、数と文字をバラバラに見ていこう!」

$$
\frac{-10x^2}{2x} = - \frac{10x^2}{2x} = -5x
$$

---

## 🔷 教材に落とし込むときの構成案(スライドやプリント)

1. **符号の組み合わせ表**
2. **分数化 → 約分の例**
3. **逆数のかけ算のイメージ(希望あれば図付き)**
4. **順序のメリット(比較例)**
5. **練習問題 → ステップ分けて書かせる**

\end{document}
