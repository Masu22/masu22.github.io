\documentclass[11pt]{article}

\usepackage[most]{tcolorbox}
\usepackage{luatexja}
\usepackage{array}

%セルの長さを指定して縦にそろえる
\newcolumntype{M}[1]{>{\centering\arraybackslash}m{#1}}

\setlength{\parindent}{0pt}
\linespread{1.2} %行間を調整

%tcolorboxのスタイルを設定
\tcbset{
  mybox/.style={
    %colback=yellow!5,
    %colframe=orange!80!black,
    fonttitle=\bfseries,
    title=#1
  }
}

\begin{document}

\begin{tcolorbox}[mybox={データの分析}]
いくつかのデータを、基準をもとに区間ごとに分ける。それを表に整理したものを\textbf{度数分布表}といい、棒グラフに整理したものを\textbf{ヒストグラム}という。\\
このとき、データを分析するための用語がいくつかある!\\

区間のことを\textbf{階級}、区間の真ん中の値を\textbf{階級値}、区間のデータの個数を\textbf{度数}という。\\

$\textbf{相対度数} = \dfrac{\text{各階級の度数}}{\text{全ての度数の合計}}$\\

はじめの階級からその階級までの度数の合計を\textbf{累積度数}という。\\

データの特徴を表す数値を\textbf{代表値}という。\\
平均値、中央値、最頻値、範囲などがある。\\

\textbf{平均値} $\cdots$ (データの数値の合計) $\div$ (データの個数)\\
\textbf{中央値} $\cdots$ データを大小順に並べたときに、真ん中のデータの値\\
\textbf{最頻値} $\cdots$ データの中で、一番多い数値\\

$\textbf{範囲} = \text{(データの最大値)} - \text{(データの最小値)}$  
\end{tcolorbox}

例えば、20人の身長を、10cmごとの区間に分けてみる。\\

%この表だけ高さ変更
\begingroup
\renewcommand{\arraystretch}{1.5}  %この表だけ行の高さを1.5倍
\begin{tabular}{|M{2.5cm}|M{1.5cm}|M{1cm}|M{3.5cm}|M{2cm}|}
\hline
階級&階級値&度数&相対度数&累積度数\\
\hline
140\ ~\ 150&145&5& $5 \div 20 = 0.25$ &5\\
\hline
150\ ~\ 160&155&10& $10 \div 20 = 0.50$ &15\\
\hline
160\ ~\ 170&165&5& $5 \div 20 =0.25$ &20\\
\hline
\end{tabular}
\endgroup

\bigskip

平均値 : $(145 \times 5 + 155 \times 10 + 165 \times 5) \div 20 = 155$\\
中央値 : 真ん中は10人目なので、155\\
最頻値 : 度数10が一番大きいので、155\\

ちなみに、階級値は、階級の両端の値をたして、2でわると求まる。\\

例えば、140~150という階級ならば、階級値は$\dfrac{140+150}{2} = 145$となる。

\end{document}