\documentclass{jsarticle}

\begin{document}

%一次方程式:まとめ(中学数学)

\section*{一次方程式とは?}

**一次方程式**とは、
変数 $x$ の**次数が1**であるような**等式**のことです。

例:

* $2x + 3 = 7$
* $-x + 5 = 0$
* $\frac{3x - 4}{2} = 1$

---

### 🔷 2. 一次方程式の基本形

$$
ax + b = 0 \quad (a \ne 0)
$$

ここで、

* $a$:変数 $x$ にかかっている係数
* $b$:定数項

このような形に変形することで、解きやすくなります。

---

### 🔷 3. 一次方程式の解法(基本ステップ)

#### 🧭 解き方の流れ(標準的手順)

1. **分数・小数をなくす(あれば)**
2. **式を整理する(移項・展開など)**
3. **$x$ の項を一方にまとめる**
4. **定数項を逆側に移項する**
5. **$x$ の係数で割る(または両辺を割る)**

---

### 🔷 4. 例題と解法

#### 🟢 例1:簡単な方程式

$$
2x + 3 = 7
$$

**解法:**

$$
2x = 7 - 3 = 4 \\
x = \frac{4}{2} = 2
$$

---

#### 🟢 例2:$x$ が両辺にある

$$
3x - 2 = x + 4
$$

**解法:**

$$
3x - x = 4 + 2 \\
2x = 6 \\
x = 3
$$

---

#### 🟢 例3:分数がある

$$
\frac{1}{2}x + \frac{1}{3} = \frac{5}{6}
$$

**解法:**
両辺に最小公倍数 $6$ をかける:

$$
6 \left(\frac{1}{2}x + \frac{1}{3} \right) = 6 \cdot \frac{5}{6} \\
3x + 2 = 5 \\
x = 1
$$

---

### 🔷 5. 一次方程式の性質

#### ✅ 解の一意性:

一次方程式は、**解が1つに定まる**(ただし例外あり)

#### ✅ 恒等式と矛盾式:

* すべての $x$ に対して成り立つ(恒等式)
   → 例:$2x + 4 = 2(x + 2)$ は常に成立 → 解は「すべての実数」
* 成立しない(矛盾式)
   → 例:$2x + 3 = 2x + 5$ → $3 = 5$ は成り立たない → 解なし

---

### 🔷 6. 指導・学習のポイント

* 「移項」は**符号を変えて移す**というルールを丁寧に教える
* 「式の変形」は算数の延長として、**計算の正確さを意識**
* 分数や小数を含む場合は、**両辺にかける操作**に慣れさせる
* **代入による検算**も習慣づける(本当に解になっているか確認)

---

### 🔷 7. よくある誤り・注意点

| 誤り例                            | 解説                      |
| ------------------------------ | ----------------------- |
| $2x + 3 = 7$ → $x = 7 - 3 = 4$ | $2x = 4$ にしてから割る必要がある   |
| 移項時に符号ミス                       | $+3$ を右辺に移すときは $-3$ になる |
| 分母のかけ忘れ                        | 分数を消すときに両辺にかけるのを忘れがち    |

---

### 🔷 8. 応用・発展につなぐために

* 一次方程式を使って**文章題を解く練習**を導入(速さ・割合など)
* **連立方程式**へつなげる(一次方程式の拡張として)
* グラフとの対応:$y = ax + b$ の**直線の交点**を求める(方程式の解と視覚的対応)

---

### 🔷 9. チェック問題(例)

#### Q1. 次の一次方程式を解け。

(1) $3x - 7 = 2x + 1$
(2) $\frac{2x - 3}{5} = \frac{x + 1}{2}$
(3) $-4x + 6 = 2 - x$

\end{document}