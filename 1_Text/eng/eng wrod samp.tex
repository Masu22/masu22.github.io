\documentclass[a4paper,12pt]{article}

\usepackage{amsmath,amssymb}
\usepackage{tcolorbox}
\usepackage{enumitem}
\usepackage[margin=25mm]{geometry}
\usepackage{titlesec}
\usepackage{multicol}
\usepackage{lmodern}
\usepackage{luatexja}
\usepackage{setspace}

\setstretch{1.15}

\titleformat{\section}[block]{\Large\bfseries}{\thesection}{1em}{}



\title{高校英語 基本動詞マスター \\ \large 第1回:重要動詞10語}
\date{}

\begin{document}

\maketitle

\section*{make}

\begin{tcolorbox}[title=語源と原義]
語源:古英語 \textit{macian}「作る・形づくる」\\
原義:素材や要素を組み合わせて何かを作る(=create / produce)
\end{tcolorbox}

\begin{tcolorbox}[title=意味の展開]

\begin{itemize}
  \item make a cake(物理的に作る)
  \item make a decision(意思を固める)
  \item make someone happy(状態を変える)
  \item make it to the station(到達する)
  \item make money(稼ぐ)
\end{itemize}

\end{tcolorbox}

\begin{tcolorbox}[title=例文]

\begin{itemize}
  \item She made a beautiful painting for her friend.
  \item Let’s make a plan before we start.
  \item He made it to the final round.
\end{itemize}

\end{tcolorbox}



\section*{take}

\begin{tcolorbox}[title=語源と原義]
語源:古英語 \textit{tacan}(北欧由来)=「掴んで持ち去る、捕まえる」\\
原義:何かを手に取って、自分のものにする(→移動・変化の始点)
\end{tcolorbox}

\begin{tcolorbox}[title=意味の展開]

\begin{itemize}
  \item take a photo(写真を撮る)
  \item take medicine(薬を飲む)
  \item take a seat(座る)
  \item take time(時間がかかる)
  \item take care of〜(世話する)
  \item take it easy(気楽にやる)
\end{itemize}

\end{tcolorbox}

\begin{tcolorbox}[title=例文]

\begin{itemize}
  \item Can you take a picture of us?
  \item I took the wrong train this morning.
  \item Please take your time.
\end{itemize}

\end{tcolorbox}



\section*{get}

\begin{tcolorbox}[title=語源と原義]
語源:古ノルド語 \textit{geta}「得る、手に入れる」\\
原義:外部から何かを得る、到達する
\end{tcolorbox}

\begin{tcolorbox}[title=意味の展開]

\begin{itemize}
  \item get a present(プレゼントをもらう)
  \item get home(家に着く)
  \item get sick(病気になる)
  \item get angry(怒る)
  \item get things done(物事を片付ける)
\end{itemize}

\end{tcolorbox}

\begin{tcolorbox}[title=例文]

\begin{itemize}
  \item I got a letter from her yesterday.
  \item He got really angry at the news.
  \item What time did you get home?
\end{itemize}

\end{tcolorbox}



\section*{give}

\begin{tcolorbox}[title=語源と原義]
語源:古英語 \textit{giefan}「与える、渡す」\\
原義:自分のものを他者に移す、譲渡する
\end{tcolorbox}

\begin{tcolorbox}[title=意味の展開]

\begin{itemize}
  \item give a gift(プレゼントを渡す)
  \item give advice(助言する)
  \item give someone a hand(手伝う)
  \item give a speech(スピーチをする)
\end{itemize}

\end{tcolorbox}

\begin{tcolorbox}[title=例文]

\begin{itemize}
  \item I gave him a book for his birthday.
  \item She gave a great speech.
  \item Can you give me a hand?
\end{itemize}

\end{tcolorbox}



\end{document}