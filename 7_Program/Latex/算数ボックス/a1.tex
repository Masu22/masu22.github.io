%Lualatexで実行できる!

\documentclass{article}
\usepackage[most]{tcolorbox}
\usepackage{xcolor}
\usepackage{luatexja}    % 日本語処理用
\usepackage{fontspec}    % フォント指定用
%\setmainfont{MS Mincho}  % Windowsの「MS 明朝」を指定

\begin{document}

% 外側のボックス:黄色背景+青枠、下側に影つき
\begin{tcolorbox}[colback=yellow, colframe=blue, boxrule=0.5mm, sharp corners, enhanced, width=\linewidth, , drop lifted shadow]

  % 内側のボックス:緑背景+白枠
  \begin{tcolorbox}[enhanced,colback=green!70!black, colframe=white, boxrule=1mm, sharp corners, width=\linewidth]
  \centering %文章を中央にそろえる!
  \color{white} \textbf{Title : Math Box}\\
  \vspace{5mm}
  {\Large \bfseries 算数用のボックス案}
  \end{tcolorbox}
\end{tcolorbox}

\end{document}

