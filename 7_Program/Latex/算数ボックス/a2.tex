\documentclass{article}
\usepackage[most]{tcolorbox}
\usepackage{xcolor}
\usepackage{lmodern}
\usepackage{tikz}
\usetikzlibrary{shadows, backgrounds}


\begin{document}
\begin{tcolorbox}[
  enhanced, % tcolorboxの拡張機能を有効化(影やオーバーレイなどが使える)
  colback=black!90, % 背景色を黒の90%(かなり暗め)に設定
  colframe=cyan,    % 枠線の色をネオン風のシアンに設定
  boxrule=1.5pt,    % 枠線の太さを1.5ptに設定
  width=\linewidth, % ボックスの幅を行幅いっぱいにする
  sharp corners,    % 角を丸めずに鋭角に設定
  drop shadow={shadow xshift=0pt, shadow yshift=0pt, color=cyan!80!black}, 
  % 影をずらさずにボックスの周囲にシアンと黒の混ざった影をつける(光っている感じ)
  left=6pt, right=6pt, top=6pt, bottom=6pt, 
  % ボックス内の上下左右の余白(パディング)を6ptずつ設定
  fontupper=\color{cyan}\bfseries\large, 
  % ボックス内テキストをシアン色、太字、大きめのフォントで表示
  overlay={% 
    % ボックスの上に重ねる装飾(光のグラデーションライン)を描く
    \begin{scope}
      % ボックスの左下から左上までの縦長グラデーション塗り
      \shade[inner color=cyan!80!white,outer color=cyan!5!black] 
        (frame.south west) rectangle ([xshift=\linewidth]frame.north west);
      % ボックス上辺の横長グラデーション(左から右へ明るい→暗い)
      \shade[left color=cyan!80!white, right color=cyan!5!black] 
        (frame.north west) rectangle ([yshift=4pt]frame.north east);
      % ボックス下辺の横長グラデーション(左から右へ暗い→明るい)
      \shade[left color=cyan!5!black, right color=cyan!80!white] 
        (frame.south west) rectangle ([yshift=-4pt]frame.south east);
    \end{scope}
  }
]

Neon gradient style box

\end{tcolorbox}



\end{document}
