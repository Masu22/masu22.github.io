\documentclass{article}
\usepackage[most]{tcolorbox}
\usepackage{xcolor}
\usepackage{lmodern}
\usepackage{tikz}
\usetikzlibrary{shadows, backgrounds}

\begin{document}

\newtcolorbox{neonGlossyBox}[1][]{
  enhanced, % tcolorboxの拡張機能を有効にする(影やオーバーレイなどが使える)
  colback=black!90, % 背景色をほぼ黒(黒の90%の濃さ)に設定
  colframe=cyan!80!white, % 枠線の色をシアン80%と白を混ぜた明るめのシアンに設定
  boxrule=1.5pt, % 枠線の太さを1.5ポイントに設定
  width=\linewidth, % ボックスの幅をページの行幅いっぱいにする
  sharp corners, % 角を丸めずに鋭角に設定
  drop shadow={shadow xshift=1pt, shadow yshift=-1pt, color=cyan!60!black}, % 影を右下にずらしてシアンと黒の混ざった色でつける
  left=6pt, right=6pt, top=6pt, bottom=6pt, % ボックス内の余白(パディング)を6ptずつに設定
  fontupper=\color{cyan}\bfseries\large, % 中のテキストをシアン色、太字、大きめフォントに設定
  overlay={ % ボックスの上に重ねる追加描画を定義
    \begin{scope}
      \shade[inner color=white!70!cyan, outer color=cyan!10!black, opacity=0.4]
        ([xshift=2pt,yshift=6pt]frame.north west) rectangle
        ([xshift=-2pt,yshift=-6pt]frame.north east);
      % ボックス上部の横長の光沢グラデーションを描画
      % 左上から右上までの長方形で、内側は明るい白とシアン混合、外側は暗めのシアン混合
      % 透明度0.4で控えめな光沢効果を表現
    \end{scope}
  },
  #1 % 引数で追加オプションを受け付ける
}

\begin{neonGlossyBox}
  Glossy neon style box with light reflection.
\end{neonGlossyBox}


\end{document}
